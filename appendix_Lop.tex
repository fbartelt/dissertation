% !TeX root = main.tex
\chapter{Properties of the L Operator}\label{app:properties-L-op}
In this chapter, we explore additional properties of the $\Lop$ operator, which is defined in \cref{def:Loperator}. First, we demonstrate that the $\Lop$ operator shares all the properties of derivatives. Let $G$ be an $m$-dimensional matrix Lie group and $\mathbf{X}\in G$ be a point in the group. Let $\SL:\mathbb{R}^m\to\mathfrak{g}$ be an isomorphism between $m$-dimensional vectors and the Lie algebra $\mathfrak{g}$ of $G$. For some function $f:G\to \mathbb{R}$, the $\Lop$ operator can be expressed as
\begin{align}
    \Lop[f](\mathbf{X})\boldsymbol{\zeta} = \frac{d}{d\varepsilon}\biggl(f\Bigl(\exp\bigl(\varepsilon\SL[\boldsymbol{\zeta}]\bigr)\mathbf{X}\Bigr)\biggr)\biggr|_{\varepsilon=0}\,\forall\,\boldsymbol{\zeta}\in\mathbb{R}^m. \label{eq:appendix-prop-Lop-definition-L}
\end{align}
\section{Linearity}
The $\Lop$ operator is linear. Specifically, for any functions $f,g:G\to \mathbb{R}$ and scalars $a,b\in\mathbb{R}$, we have
\begin{align}
    \begin{split}
        \Lop[a f + b g](\mathbf{X})\boldsymbol{\zeta} &= \frac{d}{d\varepsilon}\biggl(a f\Bigl(\exp\bigl(\varepsilon\SL[\boldsymbol{\zeta}]\bigr)\mathbf{X}\Bigr) + b g\Bigl(\exp\bigl(\varepsilon\SL[\boldsymbol{\zeta}]\bigr)\mathbf{X}\Bigr)\biggr)\biggr|_{\varepsilon=0}\\
        &= \frac{d}{d\varepsilon}\biggl(a f\Bigl(\exp\bigl(\varepsilon\SL[\boldsymbol{\zeta}]\bigr)\mathbf{X}\Bigr)\biggr)\biggr|_{\varepsilon=0} +  \frac{d}{d\varepsilon}\biggl(b g\Bigl(\exp\bigl(\varepsilon\SL[\boldsymbol{\zeta}]\bigr)\mathbf{X}\Bigr)\biggr)\biggr|_{\varepsilon=0}\\
        &= a \Lop[f](\mathbf{X})\boldsymbol{\zeta} + b \Lop[g](\mathbf{X})\boldsymbol{\zeta}\\
        &= \bigl(a \Lop[f](\mathbf{X}) + b \Lop[g](\mathbf{X})\bigr)\boldsymbol{\zeta},
    \end{split}
\end{align}
which shows that $\Lop[a f + b g] = a \Lop[f] + b \Lop[g]$, which confirms that the $\Lop$ operator is linear.
\section{Product and quotient rule}\label{sec:appendix-prop-Lop-product-quotient}
The $\Lop$ operator satisfies both the product and quotient rules. Let $f,g:G\to \mathbb{R}$ be scalar functions. By definition, we have
\begin{align}
  \begin{split}
      \Lop[fg](\mathbf{X})\boldsymbol{\zeta} =&
      \frac{d}{d\varepsilon}\biggl(f\Bigl(\exp\bigl(\varepsilon\SL[\boldsymbol{\zeta}]\bigr)\mathbf{X}\Bigr)g\Bigl(\exp\bigl(\varepsilon\SL[\boldsymbol{\zeta}]\bigr)\mathbf{X}\Bigr)\biggr)\biggr|_{\varepsilon=0}\\
      =& \frac{d}{d\varepsilon}\biggl(f\Bigl(\exp\bigl(\varepsilon\SL[\boldsymbol{\zeta}]\bigr)\mathbf{X}\Bigr)\biggr)\biggr|_{\varepsilon=0}g\Bigl(\exp\bigl(\varepsilon\SL[\boldsymbol{\zeta}]\bigr)\mathbf{X}\Bigr)\Bigr|_{\varepsilon=0}
      \\ &+ f\Bigl(\exp\bigl(\varepsilon\SL[\boldsymbol{\zeta}]\bigr)\mathbf{X}\Bigr)\Bigr|_{\varepsilon=0}\frac{d}{d\varepsilon}\biggl(g\Bigl(\exp\bigl(\varepsilon\SL[\boldsymbol{\zeta}]\bigr)\mathbf{X}\Bigr)\biggr)\biggr|_{\varepsilon=0}
      \\
      =& \frac{d}{d\varepsilon}\biggl(f\Bigl(\exp\bigl(\varepsilon\SL[\boldsymbol{\zeta}]\bigr)\mathbf{X}\Bigr)\biggr)\biggr|_{\varepsilon=0}g(\mathbf{X})
      + f(\mathbf{X})\biggl(g\Bigl(\exp\bigl(\varepsilon\SL[\boldsymbol{\zeta}]\bigr)\mathbf{X}\Bigr)\biggr)\biggr|_{\varepsilon=0}\\
      =& \Lop[f](\mathbf{X})\boldsymbol{\zeta}g(\mathbf{X}) + f(\mathbf{X})\Lop[g](\mathbf{X})\boldsymbol{\zeta}\\
        =& \bigl(\Lop[f](\mathbf{X})g(\mathbf{X}) + f(\mathbf{X})\Lop[g](\mathbf{X})\bigr)\boldsymbol{\zeta},
  \end{split}
\end{align}
since $f$ and $g$ are scalar functions. This shows that $\Lop[fg] = \Lop[f]g + f\Lop[g]$. Next, we demonstrate the quotient rule. For that, it suffices to show that $\Lop[1/g] = -\Lop[g]/g^2$. In this case, we have
\begin{align}
    \begin{split}
        \Lop\biggl[\frac{1}{g}\biggr](\mathbf{X})\boldsymbol{\zeta} =& \left.\frac{d}{d\varepsilon}\left(\frac{1}{g\Bigl(\exp\bigl(\varepsilon\SL[\boldsymbol{\zeta}]\bigr)\mathbf{X}\Bigr)}\right)\right|_{\varepsilon=0}\\
        =& -\frac{{\displaystyle\frac{d}{d\varepsilon}}\biggl(g\Bigl(\exp\bigl(\varepsilon\SL[\boldsymbol{\zeta}]\bigr)\mathbf{X}\Bigr)\biggr)\biggr|_{\varepsilon=0}}{\biggl(g\Bigl(\exp\bigl(\varepsilon\SL[\boldsymbol{\zeta}]\bigr)\mathbf{X}\Bigr)g\Bigl(\exp\bigl(\varepsilon\SL[\boldsymbol{\zeta}]\bigr)\mathbf{X}\Bigr)\biggr)\biggr|_{\varepsilon=0}}\\
        =& -\frac{\Lop[g]}{g(\mathbf{X})^2}\boldsymbol{\zeta},
    \end{split}
\end{align}
which implies
\begin{align}
    \Lop\biggl[\frac{f}{g}\biggr] = \Lop[f]\frac{1}{g} - f\frac{\Lop[g]}{g^2} = \frac{\Lop[f]g - f\Lop[g]}{g^2}.
\end{align}
Thus, the $\Lop$ operator satisfies the quotient rule.
\section{Chain rules}
A chain rule has already been shown in \cref{lemma:chainrule}, which applies to expressions of the form $\frac{d}{d\sigma}f\bigl(\mathbf{X}(\sigma)\bigr)$. In this section, we present two additional chain rules. The first pertains to the $\Lop$ operator applied to function composition, while the second concerns the partial $\Lop_\mathbf{V}$ operator for left-invariant functions.

Let $f:\mathbb{R}\to\mathbb{R}$ and $g:G\to \mathbb{R}$ be functions, and $\mathbf{X}$ be a point in $G$. The chain rule for the $\Lop$ operator of function composition is given by:
\begin{align}
    \Lop\Bigl[f\bigl(g(\mathbf{X})\bigr)\Bigr](\mathbf{X})\boldsymbol{\zeta} &= \frac{d}{d\varepsilon}\Biggl(f\biggl(g\Bigl(\exp\bigl(\SL[\boldsymbol{\zeta}]\bigr)\mathbf{X}\Bigr)\biggr)\Biggr)\Biggr|_{\varepsilon=0}\\
    &= \frac{df}{dg}\bigl(g(\mathbf{X})\bigr)\biggl(\frac{d}{d\varepsilon}g\Bigl(\exp\bigl(\SL[\boldsymbol{\zeta}]\bigr)\mathbf{X}\Bigr)\biggr)\biggr|_{\varepsilon=0}\\
    &= \frac{df}{dg}\bigl(g(\mathbf{X})\bigr)\Lop[g](\mathbf{X})\boldsymbol{\zeta},
\end{align}
which implies
\begin{align}
    \Lop\Bigl[f\bigl(g(\mathbf{X})\bigr)\Bigr](\mathbf{X}) = \frac{df}{dg}\bigl(g(\mathbf{X})\bigr)\Lop[g](\mathbf{X}).
\end{align}

We now derive a chain rule specifically related to the $\Lop_\mathbf{V}$ operator for left-invariant functions. Let $\mathbf{X},\mathbf{Y}\in G$, and $f : G \times G \to \mathbb{R}$ be a left-invariant function. Define $\mathbf{Z} = \mathbf{Y}^{-1}\mathbf{X}$, so that $f(\mathbf{X},\mathbf{Y}) = g(\mathbf{Z})$ for some $g:G\to\mathbb{R}$. Our goal is to determine the $\Lop$ operator of $g$ with respect to $\mathbf{X}$. 

First, consider a variation of $\mathbf{X}$ using the right-hand side of \eqref{eq:appendix-prop-Lop-definition-L} for $g$:
\begin{align}
    \frac{d}{d\varepsilon}\biggl(g\Bigl(\mathbf{Y}^{-1}\exp\bigl(\varepsilon\SL[\boldsymbol{\zeta}]\bigr)\mathbf{X}\Bigr)\biggr)\biggr|_{\varepsilon=0} &= \frac{d}{d\varepsilon}\biggl(g\Bigl(\mathbf{Y}^{-1}\exp\bigl(\varepsilon\SL[\boldsymbol{\zeta}]\bigr)\mathbf{Y}\mathbf{Y}^{-1}\mathbf{X}\Bigr)\biggr)\biggr|_{\varepsilon=0}\\
    &= \frac{d}{d\varepsilon}\biggl(g\Bigl(\exp\bigl(\varepsilon\mathbf{Y}^{-1}\SL[\boldsymbol{\zeta}]\mathbf{Y}\bigr)\mathbf{Y}^{-1}\mathbf{X}\Bigr)\biggr)\biggr|_{\varepsilon=0} \\
    &= \frac{d}{d\varepsilon}\biggl(g\Bigl(\exp\bigl(\varepsilon\mathcal{S}'(\boldsymbol{\zeta})\bigr)\mathbf{Y}^{-1}\mathbf{X}\Bigr)\biggr)\biggr|_{\varepsilon=0},
\end{align}
where $\mathcal{S}'(\boldsymbol{\zeta}) = \mathbf{Y}^{-1}\SL[\boldsymbol{\zeta}]\mathbf{Y}$. Since $\mathbf{Y}^{-1}\SL[\boldsymbol{\zeta}]\mathbf{Y}$ lies in $\mathfrak{g}$, there exists a vector $\boldsymbol{\zeta}'$ such that $\SL[\boldsymbol{\zeta}'] = \mathbf{Y}^{-1}\SL[\boldsymbol{\zeta}]\mathbf{Y}$. This implies that $\boldsymbol{\zeta}'$ is a function of $\boldsymbol{\zeta}$ and $\mathbf{Y}$, i.e., $\boldsymbol{\zeta}' \triangleq \boldsymbol{\zeta}'(\boldsymbol{\zeta}, \mathbf{Y})$. Therefore, we can write
\begin{align}
    \frac{d}{d\varepsilon}\biggl(g\Bigl(\mathbf{Y}^{-1}\exp\bigl(\varepsilon\SL[\boldsymbol{\zeta}]\bigr)\mathbf{X}\Bigr)\biggr)\biggr|_{\varepsilon=0} &= \frac{d}{d\varepsilon}\biggl(g\Bigl(\exp\bigl(\varepsilon\SL[\boldsymbol{\zeta}']\bigr)\mathbf{Y}^{-1}\mathbf{X}\Bigr)\biggr)\biggr|_{\varepsilon=0}\\
    &= \Lop[g](\mathbf{Z})\boldsymbol{\zeta}'(\boldsymbol{\zeta}, \mathbf{Y}) \label{appendix-prop-Lop-chain-rule-LgZ-zeta-prime},
\end{align}
by definition. Now, since $f$ is left-invariant, we can express
\begin{align}
    \Lop_\mathbf{V}[f](\mathbf{X}, \mathbf{Y})\boldsymbol{\zeta} &= 
    \Lop_\mathbf{V}[f](\mathbf{Y}^{-1}\mathbf{X}, \mathbf{I})\boldsymbol{\zeta}
    = \frac{d}{d\varepsilon}\biggl(f\Bigl(\exp\bigl(\varepsilon\SL[\boldsymbol{\zeta}]\bigr)\mathbf{Y}^{-1}\mathbf{X}, \mathbf{I}\Bigr)\biggr)\biggr|_{\varepsilon=0},
\end{align}
which leads to
\begin{align}
    \Lop_\mathbf{V}[f](\mathbf{X}, \mathbf{Y})\boldsymbol{\zeta} = \Lop[g](\mathbf{Z})\boldsymbol{\zeta}'(\boldsymbol{\zeta}, \mathbf{Y}) \label{app:prop-Lop-chain-rule-LVf-final-form},
\end{align}
which is the expression for the chain rule in the context of left-invariant functions. Clearly, $\boldsymbol{\zeta}'$ is expressed as:
\begin{align}
    \boldsymbol{\zeta}'(\boldsymbol{\zeta}, \mathbf{Y}) &= \invSL[\mathbf{Y}^{-1}\SL[\boldsymbol{\zeta}]\mathbf{Y}]\\
    &= \invSL[\sum_{i=1}^m\mathbf{Y}^{-1}\SL[\mathbf{e}_i]\mathbf{Y}\zeta_i]\\
    &= \sum_{i=1}^m\invSL[\mathbf{Y}^{-1}\SL[\mathbf{e}_i]\mathbf{Y}]\zeta_i\\
    &= \begin{bmatrix}
        \invSL[\mathbf{Y}^{-1}\SL[\mathbf{e}_1]\mathbf{Y}] & \cdots & \invSL[\mathbf{Y}^{-1}\SL[\mathbf{e}_m]\mathbf{Y}]
    \end{bmatrix} \boldsymbol{\zeta}\\
    &= \mathcal{Z}(\mathbf{Y})\boldsymbol{\zeta},
\end{align}
where the linearity properties of $\SL$ and $\invSL$ have been used. This implies that the chain rule for the $\Lop_\mathbf{V}$ operator can be expressed as:
\begin{align}
    \Lop_\mathbf{V}[f](\mathbf{X}, \mathbf{Y})\boldsymbol{\zeta} = \Lop[g](\mathbf{Z})\mathcal{Z}(\mathbf{Y})\boldsymbol{\zeta}. \label{app:prop-Lop-chain-rule-LVf-final-form-matrix}
\end{align}

\subsection{The case of SE(3)}\label{app:prop-Lop-chain-rule-SE3}
In the case of $\text{SE}(3)$, we can show that $\boldsymbol{\zeta}'(\boldsymbol{\zeta}, \mathbf{Y})$ is given by a simple matrix multiplication. Let $\SL:\mathbb{R}^6\to\mathfrak{se}(3)$ be the canonical isomorphism between $\mathbb{R}^6$ and $\mathfrak{se}(3)$, so that $\mathbf{Y}$ can be written as
\begin{align}
    \mathbf{Y} = \begin{bmatrix}
        \mathbf{R}_y & \mathbf{p}_y\\
        \mathbf{0} & 1
    \end{bmatrix},\,\mathbf{R}_y\in\text{SO}(3), \mathbf{p}_y\in\mathbb{R}^3.
\end{align}

Next, define $\widehat{\mathcal{S}}:\mathbb{R}^3\to\mathfrak{so}(3)$ as the canonical isomorphism between $\mathbb{R}^3$ and $\mathfrak{so}(3)$, and let $\boldsymbol{\zeta}=[\mathbf{v}^\top\ \boldsymbol{\omega}^\top]^\top$, where $\mathbf{v}, \boldsymbol{\omega}\in\mathbb{R}^3$. Then, we have the following expression:
\begin{align}
  \begin{split}
      \mathbf{Y}^{-1}\SL[\boldsymbol{\zeta}]\mathbf{Y} &= 
      \begin{bmatrix}
          \mathbf{R}_y^\top & -\mathbf{R}_y^\top\mathbf{p}_y\\
          \mathbf{0} & 1
      \end{bmatrix}
      \begin{bmatrix}
          \widehat{\mathcal{S}}(\boldsymbol{\omega}) & \mathbf{v}\\
          \mathbf{0} & 0
      \end{bmatrix}
      \begin{bmatrix}
          \mathbf{R}_y & \mathbf{p}_y\\
          \mathbf{0} & 1
      \end{bmatrix}\\
      &= 
      \begin{bmatrix}
          \mathbf{R}_y^\top\widehat{\mathcal{S}}(\boldsymbol{\omega})\mathbf{R}_y & \mathbf{R}_y^\top\widehat{\mathcal{S}}(\boldsymbol{\omega})\mathbf{p}_y + \mathbf{R}_y^\top\mathbf{v}\\
          \mathbf{0} & 0
      \end{bmatrix}\\
      &= \SL[\boldsymbol{\zeta}'].
  \end{split}
\end{align}
Using the properties of $\widehat{\mathcal{S}}$, we have $\mathbf{R}_y^\top\widehat{\mathcal{S}}(\boldsymbol{\omega})\mathbf{R}_y = \widehat{S}(\mathbf{R}_y^\top\boldsymbol{\omega})$, thus we can express $\boldsymbol{\zeta}'$ as
\begin{align}
    \boldsymbol{\zeta}'(\boldsymbol{\zeta}, \mathbf{Y}) = \begin{bmatrix}
        \mathbf{R}_y^\top\widehat{\mathcal{S}}(\boldsymbol{\omega})\mathbf{p}_y + \mathbf{R}_y^\top\mathbf{v}\\
        \mathbf{R}_y^\top\boldsymbol{\omega}
    \end{bmatrix}.
\end{align}

Note that $\widehat{\mathcal{S}}(\boldsymbol{\omega})\mathbf{p}_y = -\widehat{\mathcal{S}}(\mathbf{p}_y)\boldsymbol{\omega}$, so the following holds
\begin{align}
    \boldsymbol{\zeta}'(\boldsymbol{\zeta}, \mathbf{Y}) = \begin{bmatrix}
        \mathbf{R}_y^\top & -\mathbf{R}_y^\top\widehat{\mathcal{S}}(\mathbf{p}_y)\\
        \mathbf{0} & \mathbf{R}_y^\top
    \end{bmatrix}\begin{bmatrix}
        \mathbf{v}\\ \boldsymbol{\omega}
    \end{bmatrix}
    = \mathcal{Z}(\mathbf{Y})\boldsymbol{\zeta}.
\end{align}
Thus, achieving the expression \eqref{app:prop-Lop-chain-rule-LVf-final-form-matrix}:
\begin{align}
    \Lop_\mathbf{V}[f](\mathbf{X}, \mathbf{Y}) = \Lop[g](\mathbf{Z})\mathcal{Z}(\mathbf{Y}). \label{eq:appendix-prop-Lop-chain-rule-SE3}
\end{align}

\section{Connection to the Lie derivative}
Lie derivatives are directional derivatives adapted to the structure of a manifold. While standard directional derivatives rely on the properties of Euclidean space and are thus only locally valid for manifolds, Lie derivatives are defined in a way that ensures their validity globally on the manifold. Although the $\Lop$ operator pertains to functions, we begin by recalling the definition of the Lie derivative for vector fields for the sake of completeness.

In the context of vector fields, the Lie derivative provides a measure of the change of one vector field along the flow generated by another. Let $G$ be a Lie group, and let $V\in T_\mathbf{X}G$ and $W\in T_\mathbf{Y}G$ be vector fields on $G$. Since these vector fields belong to different tangent spaces, it is necessary to transport points using the flow. Let $\rho:\mathbb{R}\times G$ be the flow of $V$. Following the integral curve of $\rho$ with initial condition $\mathbf{X}$, we can evaluate $W(\rho(\sigma, \mathbf{X}))$, which lies in the tangent space of $G$ at $\rho(\sigma, \mathbf{X})$. To compare both vector fields, it is necessary to transport $W\bigl(\rho(\sigma, \mathbf{X})\bigr)$ back to $T_\mathbf{X}G$. The Lie derivative of $W$ with respect to $V$, denoted by $\mathscr{L}_V W$, at $\mathbf{X}$ is defined as follows \citep[p. 316]{Gallier2020}:
\begin{align}
    (\mathscr{L}_V W)_\mathbf{X}=\frac{d}{d\sigma}\bigl(d(\rho(-\sigma, \mathbf{X}))_{\rho(\sigma, \mathbf{X})}\bigr)\bigl(W(\rho(\sigma, \mathbf{X}))\bigr)\bigr|_{\sigma=0}.
\end{align}
This operation is often expressed more concisely using the \emph{pull-back} of $W$ by $\rho$, denoted $\rho(\sigma, \mathbf{X})^*W$, leading to the simplified expression \citep[p. 317]{Gallier2020}
\begin{align}
    (\mathscr{L}_V W)_\mathbf{X} = \frac{d}{d\sigma}\bigl(\rho(\sigma, \mathbf{X})^*W\bigr)\bigr|_{\sigma=0}.
\end{align}

The Lie derivative of a function $f:G\to \mathbb{R}$ at a point $\mathbf{X}\in G$ with respect to a vector field $V$ with flow $\rho$ is expressed as \citep[p. 121]{Lang2012}:
\begin{align}
    (\mathscr{L}_V f)_\mathbf{X} = \lim_{\sigma \to 0 }\frac{f\bigl(\rho(\sigma, \mathbf{X})\bigr) - f(\mathbf{X})}{\sigma} = \frac{d}{d\sigma}\Bigl(f\bigl(\rho(\sigma, \mathbf{X})\bigr)\Bigr)\Bigr|_{\sigma=0}.
\end{align}

For a matrix Lie group $G$, let $\mathbf{G}(\sigma)$ be a curve in $G$ with tangent vector $V$ at $\mathbf{G}(\sigma)$. The curve derivative is then given by
\begin{align}
    \frac{d}{d\sigma}\mathbf{G}(\sigma) = V(\mathbf{G}(\sigma)),
\end{align}
which, as shown in \cref{lemma:very-important-fact}, can be expressed as
\begin{align}
    \frac{d}{d\sigma}\mathbf{G}(\sigma) = \SL\bigl(\boldsymbol{\xi}(\sigma)\bigr)\mathbf{G}(\sigma),
\end{align}
for some isomorphism $\SL:\mathbb{R}^m\to \mathfrak{g}$ and function $\boldsymbol{\xi}:\mathbb{R}\to\mathbb{R}^m$.

Let $\mathbf{G} = \mathbf{G}(\sigma)$ for some fixed $\sigma$. A local flow $\rho(\varepsilon, \mathbf{G})$ of $V$ can be expressed as $\exp\bigl(\varepsilon\SL[\boldsymbol{\zeta}]\bigr)\mathbf{G}$ for small $\varepsilon$ and some vector $\boldsymbol{\zeta}\in\mathbb{R}^m$. The Lie derivative of $f$ with respect to $V$ at $\mathbf{G}$ is therefore
\begin{align}
    (\mathscr{L}_V f)_{\mathbf{G}} = \lim_{\varepsilon \to 0 }\frac{f\Bigl(\exp\bigl(\varepsilon\SL[\boldsymbol{\zeta}]\bigr)\mathbf{G}\Bigr) - f(\mathbf{G})}{\varepsilon},
\end{align}
which, by definition, is exactly equal to $\Lop[f](\mathbf{G})\boldsymbol{\zeta}$, demonstrating that the $\Lop$ operator represents a component of the Lie derivative.