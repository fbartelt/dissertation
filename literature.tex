% !TeX root = main.tex
\chapter{Literature Review} \label{chap:literature-review}
This chapter provides a literature review to contextualize our work. Two key topics sustain our research: (1) vector field guidance and (2) applications of Lie theory. We examine the relevant literature in each domain separately.

\section{Vector field guidance} \label{sec:lit-review-vector-field-guidance}
Vector field-based strategies are widely employed due to their ability to seamlessly integrate path planning, trajectory planning, and control. These steps are typically described as follows: \emph{path planning}, which involves constructing a collision-free path in the configuration space given an initial and final configuration; \emph{trajectory planning}, where a time-dependent curve is generated based on the path to serve as the robot's reference trajectory; and \emph{control}, where a control law is designed to ensure the robot closely tracks the reference trajectory \citep{Rimon1992}.

In the context of control theory, a vector field assigns a velocity vector to each point in the workspace. This vector field is constructed to guide the robot toward its goal. Consequently, the three steps above are inherently embedded in vector field guidance: the velocity vector produced by the vector field serves as the robot's control input, and integrating the robot's dynamics yields a collision-free path. The primary distinction between different vector field guidance approaches lies in the methodology used to construct the vector field. This section reviews key contributions to the field.

Initial work in vector field guidance focused on potential fields, where the vector field is derived as the gradient of a potential function. These methods typically involve: (1) an \emph{attractive potential field} to ensure convergence to the desired configuration; and (2) a \emph{repulsive potential field} to avoid collisions with obstacles. The potential field method was first introduced in \citet{Khatib1985}, in which the primary goal was to enable online collision avoidance while solving a regulation problem. However, a significant limitation of this approach is the occurrence of undesired local minima, which can trap the robot away from its goal.

To address this issue, the concept of a \emph{navigation function} was introduced in \citet{Rimon1992}. Navigation functions guarantee the existence of a single local minimum and ensure bounded torques for the robot's dynamics. However, this approach requires complete knowledge of the configuration space, and while an algorithm for constructing navigation functions is proposed by the authors, their computation remains challenging. Expanding on this work, in \citet{Conn1998} collision avoidance in environments with moving obstacles is tackled. By making minimal assumptions, the authors demonstrated that the topology of a star-shaped configuration space remains invariant during motion. This invariance allows the moving obstacle problem to be addressed globally by continuously deforming the stationary solution at a fixed time over the time interval of motion. Nevertheless, like its predecessor, this method necessitates full knowledge of the configuration space. These potential field-based approaches are considered classical methodologies in vector field guidance, as discussed in foundational texts such as \citet[p. 77]{Choset2005} and \citet[p. 299]{Spong2020}.

In multi-agent scenarios, achieving a specific geometric pattern, or desired formation, is often a primary objective \citep{Chaimowicz2005,Mong-yingA.Hsieh2006,Pimenta2007}. For example, \vthree{the strategy proposed in} \citet{Chaimowicz2005} relies on constructing patterns using interpolated radial basis functions, where the resulting vector field was derived as the gradient of this interpolation. Individual agents were assigned repulsive fields to ensure proper distribution and avoid overlap. Building on this, in \citet{Mong-yingA.Hsieh2006} \vthree{this strategy was refined} by improving collision avoidance between agents and providing convergence proofs for the formation. In a different approach inspired by fluid dynamics, a method for generating two-dimensional geometric patterns for robot swarms was developed \citep{Pimenta2007}. By modeling agent interactions using principles of hydrodynamics, the vector field was computed through finite element solutions of Laplace's equation, akin to simulating fluids in electrostatic fields. Besides providing a decentralized control of the formation, this method also incorporated static obstacle avoidance.

A more general problem is addressed in \citet{goncalves2010vectorfield}, which proposes a vector field strategy for robot navigation along time-varying curves in $n$-dimensional Euclidean space. Although this also applies to formation problems, it encompasses both path and trajectory tracking. The target curve is implicitly defined as the intersection of zero-level surfaces. Specifically, a one-dimensional curve $\mathcal{C}$ embedded in $\mathbb{R}^n$ is described as the intersection of $(n-1)$-dimensional surfaces, with level sets are $\alpha_i:\mathbb{R}^n\times\mathbb{R}_+\to\mathbb{R}$, such that $\mathcal{C} = \{ [ \mathbf{x}^\top,\ t]^\top | \alpha_i(\mathbf{x}, t) = 0\,\forall\,i\le n-1 \}$. However, computing these surfaces remains a challenging task, and this methodology may still suffer from singularities.

An extension of this work is presented in \citet{yao2021singularity}, which introduces a singularity-free vector field capable of handling self-intersecting curves. These improvements are achieved by projecting the curve into a higher-dimensional space. Despite these advances, challenges related to computing zero-level surfaces persist. Further, in \citet{yao2022topological}, the method of \citet{goncalves2010vectorfield} is generalized to systems embedded in smooth Riemannian manifolds. A notable contribution of this work is the topological analysis of convergence and stability for the vector field methodology in such contexts. It is proven that global convergence in $n$-dimensional Euclidean space is unattainable, as it inevitably leads to at least one singular point.

Curves are often more intuitive and easier to describe using parametric representations, a concept also explored in vector field guidance strategies. For example, parametric representations of static, possibly self-intersecting curves, alongside contraction theory, were employed in \citet{Wu2018} to develop vector field strategies for systems embedded in manifolds. In \citet{Rezende2022}, parametric representations of curves were applied to address time-varying curves in $n$-dimensional Euclidean space, with the gradient of the Euclidean distance guiding the robot toward the curve. This approach was further extended in \citet{Nunes2022} to incorporate collision avoidance with static and dynamic obstacles.

In this work, the vector field strategy from \citet{Rezende2022} is generalized for application to matrix Lie groups. A detailed review of this approach is presented in \cref{sec:adriano-review} to facilitate connections between our generalization and the original methodology. Key properties of the approach that ensure convergence and circulation around the target curve are outlined, forming the basis for our generalization. Notably, only static curves are considered, thereby excluding the time-varying aspect of the original work. The proposed generalization is also comparable to the work in \citet{yao2022topological}, which focuses on smooth Riemannian manifolds -- structures that encompass matrix Lie groups. This renders their work broader in scope compared to ours. However, this broader approach does not exploit the group structure, which can eliminate redundant control inputs. For example, in the case of the manifold $\text{SO}(3)$ of rotation matrices, the control input is expressed as the time derivative of a rotation matrix, resulting in a dimension of $9$. In contrast, leveraging the group structure reduces the control input to a dimension of $3$, corresponding to the angular velocity. Thus, our work is more specific and arguably more intuitive. Another advantage of \vthree{our} approach is its use of parametric representations of curves, which are simpler to handle compared to the implicit representations in \citet{yao2022topological}.

\section{Lie theory applications} \label{sec:lit-review-lie-theory}
Brief allusions to Lie theory are common in control theory and robotics literature. Yet, most books and papers seldom explore what makes Lie groups and Lie algebras truly remarkable. As a result, readers might easily assume that terms like `Lie derivatives', `Lie group', or `special orthogonal group' are merely naming conventions. To call something a `Lie group' is to endow it not merely with a name, but with a wealth of profound properties that underpin its remarkable utility. In this section, a spectrum of applications across diverse fields is explored, unveiling the profound elegance and utility of this theory.

In \citet{Murray1994}, the usage of Lie groups and Lie algebras is extensive, ranging from the definition of kinematic chains to Euler-Lagrange equations. Although more focused on rigid body motions, the text introduces the concepts of Lie theory, and much of the underlying mathematics of robot motion depends on understanding these concepts. Geometric control theory shares connections with Lie theory since the former is based on differential geometry. As such, in \citet{Bullo2004}, a more in-depth study on Lie groups and Lie algebras is presented, relying heavily on Lie derivatives and Lie brackets to develop the theory of geometric control.

In the scope of control theory, different applications using the language of Lie groups are found. Matrix Lie groups have been used in invariant extended Kalman filters and to prove a strong property of asymptotic convergence around any trajectory of the system \citep{Barrau2017}. In \citet{Mccarthy2020}, global synchronization of a network of agents in a matrix Lie group is achieved. The control law is smooth, distributed, nonlinear, and discrete, and for specific groups, it is shown to guarantee exponential rate synchronization. A hybrid control scheme for trajectory tracking on $\text{SO}(3)$ and $\text{SE}(3)$ is proposed in \citet{Wang2022}. This scheme uses a potential function on an augmented space to guarantee global asymptotic stability. In \citet{Duong2024}, the Hamiltonian formulation of robot dynamics is generalized to a port-Hamiltonian formulation on matrix Lie groups. This generalization is then used to design a neural ordinary differential equation model to enable data-driven learning of robot dynamics. Furthermore, a control law is proposed to ensure trajectory tracking based on the learned dynamics.

More focused on the computer science field, Lie theory appears in many works on machine learning and pattern recognition. Lie theory has been used to simplify the real-time three-dimensional visual tracking of complex objects \citep{Drummond2002}. This methodology has also been showcased to enable visual-servoing of a robot manipulator. In \citet{Vemulapalli2014}, a novel representation for human skeletons is proposed using a Lie group of many direct products of $\text{SE}(3)$. This representation is shown to perform better than the state of the art for human action recognition. The survey in \citet{Lu2020} highlights the advantages of the innovative learning paradigm of Lie group machine learning, as well as its application to image processing and neuromorphic synergy learning.

Lie theory has found extensive applications across diverse fields of study. For instance, in numerical analysis, the Butcher group -- a Lie group -- is useful for studying ordinary differential equations via the Runge-Kutta method \citep{Bogfjellmo2017}. In Hamiltonian mechanics, it has been employed to analyze complex dynamical systems, as demonstrated in \citet{Hamburger2009}, with applications including the harmonic oscillator and Kepler's problem. Within quantum physics, Lie groups help explain unusual transport phenomena observed in physical systems \citep{Ilievski2021}. The Lorentz group and the special orthogonal group feature prominently in discussions of quantum gravity \citep{Dreyer2003,Kapec2017}.
