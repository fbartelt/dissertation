% !TeX root = main.tex
\chapter{Literature Review} \label{chap:literature-review}
In this chapter, we present a literature review to contextualize our work. Three major topics corroborate our research: (1) vector field guidance, (2) adaptive control, and (3) applications of Lie theory. We will examine relevant literature in each of these areas separately.

\section{Vector Field Guidance} \label{sec:lit-review-vector-field-guidance}
The vector field-based strategy is widely used due to its ability to integrate path planning, trajectory planning, and control. These steps are typically outlined as follows: (1) given an initial and final configuration, a collision-free path is constructed in the configuration space, a process known as \emph{path planning}; (2) based on this path, a time-dependent curve is generated to serve as a reference trajectory for the robot, referred to as \emph{trajectory planning}; (3) a control law is designed to ensure the robot closely tracks the reference trajectory \citep{Rimon1992}.

In the context of control theory, a vector field assigns a velocity vector to each point in the workspace. This vector field is constructed such that the velocity vector at each point guides the robot toward its goal. Consequently, the three aforementioned steps are inherently embedded in vector field guidance: the velocity vector produced by the vector field serves as the robot's control input, and integrating the robot's dynamics yields a collision-free path. The primary distinction between different approaches in vector field guidance lies in the methodology used to construct the vector field. This section reviews key contributions in this area.

Early developments in vector field guidance focused on potential fields, where the vector field is derived as the gradient of a potential field. These methods typically define an attractive potential field to ensure convergence to the desired configuration and a repulsive potential field to avoid collisions with obstacles. The potential field method was first introduced in \citet{Khatib1985}, in which the primary goal was to enable online collision avoidance while solving a regulation problem. However, a major drawback of this approach is the inherent occurrence of undesired local minima.

To address this issue, the concept of a \emph{navigation function} was introduced in \citet{Rimon1992}. A navigation function is a specific type of potential function that guarantees the existence of only one local minimum. Moreover, it ensures bounded torques for the robot's dynamics. Despite these advantages, the approach requires complete knowledge of the configuration space, and although the authors propose an algorithm for this purpose, computing navigation functions remains non-trivial in general. Building on this work, in \citet{Conn1998} a solution is proposed for collision avoidance in environments with moving obstacles. By making minimal assumptions, the authors demonstrated that the topology of the star-shaped configuration space remains invariant over the time interval of motion. This invariance allows the moving obstacle problem to be addressed globally through a continuous deformation of the solution to the stationary problem at a fixed time within the interval. However, as this approach relies on the methodology of \citet{Rimon1992} to solve the stationary problem, it similarly requires full knowledge of the configuration space. These potential field-based approaches are well-established and are considered classical methodologies in vector field guidance, as highlighted in foundational texts such as \citet[p.77]{Choset2005} and \citet[p. 299]{Spong2020}.

In certain applications, it is desirable for a group of robots to converge to a specific geometric pattern, i.e., achieve a desired formation \citep{Chaimowicz2005,Mong-yingA.Hsieh2006,Pimenta2007}. This can be viewed as a specialized regulation problem and has been extensively studied. For instance, in \citet{Chaimowicz2005} the authors proposed constructing a desired pattern using the interpolation of multiple radial basis functions, with the resulting vector field derived from the gradient of this interpolation. To ensure proper distribution of robots within the pattern, each agent is assigned a repulsive field that acts on its neighbors. This work was later extended by \citet{Mong-yingA.Hsieh2006}, who adopted a similar approach but introduced improvements in handling collisions between agents. These modifications also allowed the authors to provide convergence proofs for the formation. Drawing inspiration from the behavior of fluids in electrostatic fields, \citet{Pimenta2007} developed a method to generate two-dimensional geometric patterns for robot swarms. This work treated the interactions between agents using principles of hydrodynamics, enabling decentralized control of the formation. The vector field was computed by solving Laplace's equation using the finite element method, analogous to simulating fluids in electrostatic fields. This approach also allowed for the inclusion of static obstacles in the workspace.

A related but more general problem is addressed in \citet{goncalves2010vectorfield}, in which a vector field strategy is proposed for robot navigation along time-varying curves in the $n$-dimensional Euclidean space. Although this also applies to formation problems, it encompasses both path and trajectory tracking. The target curve is defined implicitly by the intersection of zero-level surfaces. Specifically, a one-dimensional curve $\mathcal{C}$ embedded in $\mathbb{R}^n$ is described as the intersection of $(n-1)$-dimensional surfaces, whose level sets are $\alpha_i:\mathbb{R}^n\times\mathbb{R}_+\to\mathbb{R}$, such that $\mathcal{C} = \{ [ \mathbf{x}^\top,\ t]^\top | \alpha_i(\mathbf{x}, t) = 0\,\forall\,i\le n-1 \}$. However, computing these surfaces is non-trivial, and this vector field methodology may still suffer from the presence of singularities. An extension of this work is presented in \citet{yao2021singularity}, in which a singularity-free vector field is proposed that is also capable of handling self-intersecting curves. These two main properties are achieved by projecting the curve into a higher-dimensional space. Despite these improvements, this approach still faces challenges related to the computation of zero-level surfaces. Furthermore, based on \citet{goncalves2010vectorfield}, a generalization to systems embedded in smooth Riemannian manifolds is introduced in \citet{yao2022topological}. A significant contribution in this work is the topological analysis of convergence and stability for the vector field methodology in this context. It is proven that global convergence in the $n$-dimensional Euclidean space is impossible, inevitably leading to at least one singular point.

Curves are generally more intuitive and easier to describe when a parametric representation is used. This concept has also been explored in constructing vector field guidance strategies. For example, parametric representations of static, possibly self-intersecting curves and contraction theory have been applied to formulate vector field strategies for systems embedded in manifolds \citep{Wu2018}. In \citet{Rezende2022}, parametric representations of curves are used to address time-varying curves in the $n$-dimensional Euclidean space, where the gradient of the Euclidean distance is utilized to guide the robot toward the curve. Moreover, this approach has been extended in \citet{Nunes2022} to incorporate collision avoidance with static and dynamic obstacles.

In this work, we build upon \citet{Rezende2022} to generalize the vector field strategy for application to matrix Lie groups. A detailed review of this approach is presented in \cref{sec:adriano-review} to facilitate connections between our generalization and the original methodology. This review outlines the key properties of the approach that ensure convergence and circulation around the target curve, which serve as the basis for our generalization. Notably, we only consider static curves, thereby excluding the time-varying aspect of the original work. Our generalization is also comparable to \citet{yao2022topological}, which focuses on smooth Riemannian manifolds -- structures that encompass matrix Lie groups. This renders their work broader in scope compared to ours. However, this broader approach does not exploit the group structure, which can eliminate redundant control inputs. For example, in the case of the manifold $\text{SO}(3)$ of rotation matrices, the control input is expressed as the time derivative of a rotation matrix, resulting in a dimension of $9$. In contrast, leveraging the group structure reduces the control input to a dimension of $3$, corresponding to the angular velocity. Thus, our work is more specific and arguably more intuitive. Another advantage of our approach is its use of parametric representations of curves, which are simpler to handle compared to the implicit representations in \citet{yao2022topological}.
% \section{Adaptive Control} \label{sec:lit-review-adaptive-control}

\section{Lie Theory Applications} \label{sec:lit-review-lie-theory}
Brief allusions to Lie theory are common in control theory and robotics literature. Yet, most books and papers seldom explore what makes Lie groups and Lie algebras truly remarkable. As a result, readers might easily assume that terms like `Lie derivatives', `Lie group', or `special orthogonal group' are merely naming conventions. To call something a `Lie group' is to endow it not merely with a name, but with a wealth of profound properties that underpin its remarkable utility. In this section, we explore a spectrum of applications across diverse fields, unveiling the profound elegance and utility of this theory.

In \citet{Murray1994} the usage of Lie groups and Lie algebras are extensive, ranging from the definition of kinematic chains to Euler-Lagrange equations. Although more focused on rigid body motions, the authors introduce the concepts of Lie theory and many of the underlying math of robot motion in the text depends on the understanding of these concepts. Geometric control theory shares connections with Lie theory, since the former is based on differential geometry. As such, in \citet{Bullo2004} the authors present a more indepth study on Lie groups and Lie algebras, relying heavily on Lie derivatives and Lie brackets to develop the theory of geometric control. 

In the scope of control theory, different applications are found using the language of Lie groups. Matrix Lie groups have been used in invariant extended Kalman filters, and used to prove a strong property of asymptotic convergence around any trajectory of the system \citep{Barrau2017}. In \citet{Mccarthy2020}, the global synchronization of a network of agents in a matrix Lie group is achieved. The control law is smooth, distributed, nonlinear and discrete, and for specific groups, it is shown to guarantee exponential rate synchronization. A hybrid control scheme for trajectory tracking on $\text{SO}(3)$ and $\text{SE}(3)$ is proposed in \citet{Wang2022}. This scheme utilizes a potential function on a augmented space to guarantee global asymptotic stability. In \citet{Duong2024}, the Hamiltoninan formulation of robot dynamics is generalized to a port-Hamiltonian formulation on matrix Lie groups. This generalization is then used to design a neural ordinary differential equation model to enable data-driven learning of robot dynamics. Furthermore, the authors also propose a control law that ensures trajectory trakcing based on the learned dynamics.

More focused on the computer science field, Lie theory appears in many works on machine learning and pattern recognition. Lie theory was used to simplifify the real-time three-dimensional visual tracking of complex objects \citep{Drummond2002}. This methodology has also been showcased to enable visual-servoing of a robot manipulator. In \citet{Vemulapalli2014}, a novel representation for human skeletons is proposed using a Lie group of many direct products of $\text{SE}(3)$. This representation is shown to perform better than the state of the art for human action recognition. The survey in \citet{Lu2020} highlights the advantages of the innovative learning paradigm of Lie group Machine Learning, as well as its application to image processing and neuromorphic synergy learning.

Lie theory has found extensive applications across diverse fields of study. For instance, in numerical analysis, the Butcher group -- a Lie group -- is useful for studying ordinary differential equations via the Runge-Kutta method \citep{Bogfjellmo2017}. In Hamiltonian mechanics, it has been employed to analyze complex dynamical systems, as demonstrated in \citet{Hamburger2009}, with applications including the harmonic oscillator and Kepler's problem. Within quantum physics, Lie groups help explain unusual transport phenomena observed in physical systems \citep{Ilievski2021}. The Lorentz group and the special orthogonal group feature prominently in discussions of quantum gravity \citep{Dreyer2003,Kapec2017}.

% Other works deal with vector fields in a more general setting. For example, vector fields in Lie groups have been developed in learning contexts in \citet{Lin2009} and \citet{urain2022learning}. The work in \citet{Akhtar2022} presents a vector field approach for controlling an element of a Lie group to achieve robust convergence to a desired target. 
