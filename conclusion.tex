% !TeX root = main.tex
\chapter{Conclusion and Future Work}\label{ch:conclusion}
In this work, we presented a strategy for generating vector fields on connected matrix Lie groups that enforce convergence to and circulation around curves defined within the same group. This approach generalizes our previous work, which was limited to parametric curve representations in Euclidean space. To achieve this broader applicability, we examined key properties of distance functions that maintain the desired features of the vector field approach. The three essential properties -- left-invariance, chainability, and local linearity -- not only support this generalization to matrix Lie groups but also permit flexibility in metric choice for Euclidean space applications. It is noteworthy that, since we use an isomorphism between the euclidean space and the Lie algebra, the proposed strategy also employs a non-redundant control input, meaning that it acts on the group's degrees of freedom directly, without the need for specifying elements of the tangent space as inputs.

We validated our strategy within the special Euclidean group $\text{SE}(3)$ under a kinematic control approach, and provided an efficient algorithm for computing the vector field in this context. The results of the kinematic control showcase the convergence and circulation proofs for the vector field, and demonstrate the ability to generate complex motions in 3D space. We also showed that the vector field can be used as a high-level controller that acts as a velocity reference for a lower-level dynamic controller. Specifically, we showed that the usage of the vector field guidance in an adaptive control scenario renders an autonomous system capable of tracking the target curve with high precision, even in the presence of unknown parameters. The dynamic control scenario also employs the usage of the group $\text{ISE}(3)$ of independent rigid motions, which showcases the generality of the proposed strategy.

For both these simulations, we employed two different methods for the computation of the tangent component, one nummeric and one analytical, and showed that both methods yield satisfactory results. Although we did not use it in simulations, we also presented a method for computing the normal component explicitly, which can be used in future works to improve the tracking performance of the system, since the approximation approach used in this work could lead to nummeric errors.

Future work will explore extensions to time-varying curves and investigate simpler distance functions with the required properties as in \cref{thm:convergence-vector-field}. Since the bottleneck of the proposed strategy is the computation of the distance, searching for optimization algorithms capable of handling the minimization of functions in matrix Lie groups is also of interest. Validation of this strategy in a real omnidirectional UAV is also of interest, along with the consideration of self-intersecting curves, which would enable more complex motions, such as lemniscate-like trajectories. It would also be valuable to explore which conditions are assured when the smoothed distance is used. Finally, the employed adaptive control strategy could be further improved by also considering uncertainties in the agents and embracing the agents' dynamics.