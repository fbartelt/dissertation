% !TeX root = main.tex
\chapter{Conclusion and Future Work}\label{ch:conclusion}
In this work, we presented a strategy for generating vector fields on connected matrix Lie groups that enforce convergence to and circulation around curves defined within the same group. This approach generalizes our previous work, which was limited to parametric curve representations in Euclidean space. To achieve this broader applicability, we examined key properties of distance functions that maintain the desired features of the vector field approach. The three essential properties -- left-invariance, chainability, and local linearity -- not only support this generalization to matrix Lie groups but also permit flexibility in the choice of metric for Euclidean space applications. Notably, since we use an isomorphism between the Euclidean space and the Lie algebra, the proposed strategy also employs a non-redundant control input, meaning that it acts on the group's degrees of freedom directly, eliminating the need to specify elements of the tangent space as inputs.

We validated our strategy within the special Euclidean group $\text{SE}(3)$ under a kinematic control approach, and provided an efficient algorithm for computing the vector field in this context. The kinematic control results validate the convergence and circulation proofs for the vector field, and demonstrate the ability to generate complex motions in 3D space. \vtwo{We also demonstrated the generality of the strategy by applying kinematic control on the group $\text{SO}^+(3, 1)$, showcasing  its capability to control Lorentz transformation matrices.} 

Additionally, we showed that the vector field can serve as a high-level controller, acting as a velocity reference for a lower-level dynamic controller. In particular, we demonstrated that using the vector field guidance in an adaptive control scenario enables an autonomous system to track the target curve with high precision, even in the presence of unknown parameters. The dynamic control scenario also involves the use of the group $\text{ISE}(3)$ of independent rigid motions, highlighting the generality of the proposed strategy.

For these simulations, we employed two methods for computing the tangent component -- one numerical and one analytical -- and showed that both methods yield satisfactory results. Although not used in the simulations, we also introduced a method for explicitly computing the normal component, which could improve the tracking performance in future work, as the approximation approach used here may introduce numerical errors, making the explicit normal component a valuable tool.

Future work will focus on extending this strategy to time-varying curves and investigating simpler distance functions with the necessary properties, as outlined in \cref{thm:convergence-vector-field}. Given that the computation of the distance is the primary bottleneck in the proposed strategy, exploring optimization algorithms for minimizing functions on matrix Lie groups will be an area of interest. Additionally, validating this strategy on a real omnidirectional UAV is a promising avenue, as is the consideration of self-intersecting curves, which would enable more complex motions, such as lemniscate-like trajectories. The use of smoothed distances will also be examined to determine the conditions under which they ensure the desired properties. \vthree{Addressing nonholonomic constraints is also of interest, as it would allow for the control of UAVs that do not have omnidirectional capabilities.} Lastly, the adaptive control strategy could be further refined by incorporating agent uncertainties and dynamics.