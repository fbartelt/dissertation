% !TeX root = main.tex
\chapter{Introduction}\label{chap:Intro}

Introduction and motivation of the work.

\section{Objective and Contributions}

Write here the main goals and contributions of this work.

Although it does not improve the understanding of this work, we add here a disclaimer to try to answer a possible question ``Why does this work involves adaptive control if the main focus is kinematic control?''. The work here first began through a search on what could be done with the knowledge acquired in the gradute courses, focusing on dynamic control applying nonlinear techniques. The main idea was to go a step further into what was developed in the courses assignments. After some research, we found another interesting paper \citep{Culbertson2021}, that treated an adaptation problem for the trajectory tracking of a manipulated body. Since the authors considered a trajectory tracking, the first question was ``can we change this to a vector field approach, thus changing the trajectory tracking to path tracking?''. As the answer was positive, what will be shown in the following text, we were faced with two choices: (1) develop more on the dynamic adaptive control, or (2) develop more on the kinemtic control, generalizing the vector field approach. The second option was chosen, as it aligns more with the author's interests and would open more possibilities for future works. Under this consideration, we extended the vector field approach in \citet{Rezende2022} to account for orientations, and thus developed a kinematic control strategy to work in $\mathbb{R}^3\times\text{SO}(3)$. Thus, our first published work covers both the usage of vector fields for path tracking in the adaptive scheme, and the generalization of the vector field approach to work in $\mathbb{R}^3\times\text{SO}(3)$, as it is published \citep{Pessoa2024}. The properties derived in this work gave us an insight that there existed more general properties for the vector field guidance, thus another question had risen ``How far can we extend this and maintain some of the intuitiveness of \citet{Rezende2022}?''. This question led to the development of a more general vector field guidance strategy, that works for systems represented as matrix Lie groups. This work was submitted to Automatica and is currently under review.

The previous disclaimer not only provides a ``fun fact'' on how this work was developed, but also justifies modifications made in this text. Here, we will focus on the more generalized version of the vector field approach, that works for systems represented as matrix Lie groups, and treat the work in \citet{Pessoa2024} as an application of this strategy. For this reason, we present our results in a non-chronological order, as we believe it is more intuitive to present the more general case first, and then show how it can be applied to a specific case. We have rewritten some of the text in \citet{Pessoa2024} to better fit the context of this work, and so it will differ from the original text. We also present the results of the collaborative simulations in this text, as we believe it is more intuitive to present the results of the application of the strategy after the strategy itself. In addition, we also dive deeper into some derivations and proofs of both works.

\textbf{TODO}
\begin{itemize}
    \item Add explicit derivative for $\widehat{D}$;
    % \item Improve Lie group section;
    % \item Add Lie algebra section;
    % \item Rerun collaborative simulations using the modified model;
    \item Change proposition 4.23 A,B etc. to X, Y...
    \item Run kinematic sim for other Lie group;
    % \item Add introduction and literature review;
    \item Add introduction
    \item Change symbols to reduce complexity;
    \item Improve proof for the derivative of Lie group element, based on flows;
    \item Add Flows + integral curves subsection in Lie group section;
    \item Maybe change spheres in trajectory plots to UAV models.
    \item Add image showing the interpolated curve?
\end{itemize}

\section{Structure of the Text}

Write here the dissertation structure.

