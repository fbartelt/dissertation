% !TeX root = main.tex
\chapter{Kinematic Control}\label{ch:kinematic}
The proposed vector field strategy was developed for any connected matrix Lie group. In this \vtwo{chapter}, we define a path and EE-distance function specifically for a class of groups known as exponential Lie groups (see \cref{sec:background-exponential-map}), which are among the most common in engineering applications.

Our objective is then providing the necessary tools to control a system $\mathbf{H}(t)$ in some exponential Lie group $G$ as the following diagram illustrates:

% A Lie group $G$ is termed \emph{exponential} if the exponential map $\exp:\mathfrak{g}\to G$ is surjective \citep{djokovic1995exponential}, i.e., for each $\mathbf{Z}\in G$ the equation $\exp(\mathbf{Y}) = \mathbf{Z}$ has at least one solution $\mathbf{Y}\in\mathfrak{g}$. Examples of such groups include the special orthogonal group $\text{SO}(n)$ \citep[p. 28]{Gallier2020}, the special Euclidean  group $\text{SE}(n)$ \citep[p. 42]{Gallier2020}, and the Heisenberg group $\text{H}$ \citep[p. 75]{Hall2015}. Additional non-trivial examples can be found in \citet{djokovic1995exponential}. In contrast, examples of non-exponential Lie groups include the special linear group $\text{SL}(n)$ \citep[p. 28]{Gallier2020}.






% Thus, it suffices to show how to compute $\frac{\partial \widehat{E}}{\partial \mathbf{Z}_i}$. This can be taken as the $i^{th}$ row of the $4 \times 4$ matrix:
% \begin{equation}
%     \frac{\partial \widehat{E}}{\partial \mathbf{Z}}(\mathbf{Z}) \triangleq \left[\begin{array}{cc} \frac{\partial \widehat{E}}{\partial \mathbf{Q}}( \mathbf{Z}) & \mathbf{0} \\
%     \frac{\partial \widehat{E}}{\partial \mathbf{t}}( \mathbf{Z}) & 0 \end{array}\right]
% \end{equation}
% \noindent in which, as in Algorithm 1, $\mathbf{Q}$ and $\mathbf{t}$ are the rotation matrix and 3D translation vectors of $\mathbf{Z}$, respectively,  $ \frac{\partial \widehat{E}}{\partial \mathbf{Q}}$ the $3 \times 3$ matrix in which the entry in row $i$ and column $j$ is $\frac{\partial \widehat{E}}{\partial Q_{ji}}$ and $\frac{\partial \widehat{E}}{\partial \mathbf{t}}$ the $1 \times 3$ row vector in which the column $j$ is $\frac{\partial \widehat{E}}{\partial t_j}$. The algorithms for computing the two elements of this matrix is as follows (in which $u, v, \theta$, $\alpha$ and $\mathbf{M}$ were defined in Algorithm 1):

% \begin{itemize}
%     \item Compute 
%     \begin{eqnarray}
%        &&  \alpha'(\theta) = \frac{ \theta {+} (1 {-}\theta^2) v{-} (\theta+v)u}{2(u-1)^3} \ , \ \mathbf{N} {\triangleq} \frac{\mathbf{Q}{+}\mathbf{Q}^\top}{2}{-}\mathbf{I} \nonumber \\
%        && \frac{\partial \theta}{\partial \mathbf{Q}} = \frac{u}{4v}(\mathbf{Q}^{\top}-\mathbf{Q}) - \frac{v}{2}\mathbf{I}.
%     \end{eqnarray}
%     \item Finally, compute 
%     \begin{eqnarray}
%         && \frac{\partial \widehat{E}}{\partial \mathbf{Q}} = \frac{1}{\widehat{E}}\Bigg(\big(2\theta + \alpha' \mathbf{t}^{\top}\mathbf{N}\mathbf{t} \big) \frac{\partial \theta}{\partial \mathbf{Q}} + \alpha \mathbf{t}\mathbf{t}^{\top}\Bigg). \nonumber \\
%         && \frac{\partial \widehat{E}}{\partial \mathbf{t}} = \frac{1}{\widehat{E}} \mathbf{t}^{\top} \mathbf{M} .
%     \end{eqnarray}
% \end{itemize}
% \end{comment}