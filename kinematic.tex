% !TeX root = main.tex
\chapter{Kinematic Control}\label{ch:kinematic}
The proposed vector field strategy was developed for any connected matrix Lie group. In this chatper, we define a path and EE-distance function specifically for a class of groups known as \emph{exponential} Lie groups, which are among the most common in engineering applications.

Our objective is then providing the necessary tools to control a system $\mathbf{H}(t)$ in some exponential Lie group $G$ as the following diagram illustrates:
\begin{figure}[ht]
    \centering
    \begin{tikzpicture}[
        block/.style={draw, rectangle, minimum height=1.2cm, minimum width=2.4cm, align=center},
        arrow/.style={->, >=stealth, thick},
        label/.style={font=\small}
    ]
    
    % Nodes
    \node[block] (controller) {Vector Field};
    \node[block, right=2cm of controller] (plant) {System};
    % \node[block, below=1cm of plant] (estimation) {Estimation of $\theta_c$};
    
    % Arrows between blocks
    \draw[arrow] (controller) -- node[above, label] {$\Psi(\mathbf{H})$} (plant);
    % Input and Output Arrows
    \node[left=1.5cm of controller, yshift=0.25cm] (input) {Curve $\mathcal{C}\in G$};
    \draw[arrow] (input) -- ($(controller.west)+(0,0.25)$);    
    \node[right=1.5cm of plant] (output) {$\mathbf{H}(t)\in G$};
    \draw[arrow] (plant.east) -- (output);    
    % Feedback loop
    % \draw[arrow] ($(plant.east)+(0.75,0)$) |- ++(0,-1.5) -| (aux) -| ($(controller.west)+(0,-0.25)$);
    \draw[arrow] 
    ($(plant.east)+(0.75,0)$) |- ++(0,-1.25)
    -|  ($(controller.west)+(-0.75,-0.25)$)                          
    -- ($(controller.west)+(0,-0.25)$);

    \end{tikzpicture}
    \caption{Block diagram of the kinematic control using the vector field guidance.}
    \label{fig:kinematic-control-diagram}
\end{figure}
% A Lie group $G$ is termed \emph{exponential} if the exponential map $\exp:\mathfrak{g}\to G$ is surjective \citep{djokovic1995exponential}, i.e., for each $\mathbf{Z}\in G$ the equation $\exp(\mathbf{Y}) = \mathbf{Z}$ has at least one solution $\mathbf{Y}\in\mathfrak{g}$. Examples of such groups include the special orthogonal group $\text{SO}(n)$ \citep[p. 28]{Gallier2020}, the special Euclidean  group $\text{SE}(n)$ \citep[p. 42]{Gallier2020}, and the Heisenberg group $\text{H}$ \citep[p. 75]{Hall2015}. Additional non-trivial examples can be found in \citet{djokovic1995exponential}. In contrast, examples of non-exponential Lie groups include the special linear group $\text{SL}(n)$ \citep[p. 28]{Gallier2020}.
\section{Path and EE-Distance Function for Exponential Lie Groups}\label{sec:kinematic-path-ee-dist-exp-group} % \label{subs:explconst}
For exponential groups, while the exponential map is surjective, it is not necessarily bijective. However, we can define an inverse function in the following manner. Let $\mathbb{R}_+^{n\times n}$ be the set of real matrices with positive eigenvalues. Additionally, let $\mathcal{L}^n$ represent the set of $n\times n$ real matrices whose eigenvalues $\lambda$ lie within the strip $\{\lambda : -\pi < \text{Im}(\lambda) < \pi\}$. The function $\exp:\mathcal{L}^n\to\mathbb{R}_+^{n\times n}$ is bijective, ensuring the existence of an inverse function $\Log: \mathbb{R}_+^{n\times n} \to \mathcal{L}^n$, referred to as the \emph{principal logarithm} \citep[p. 319]{Gallier2020}. Furthermore, let $\mathcal{L}_{\mathfrak{g}}^n\subseteq\mathfrak{g}$ be the subset of $n \times n$ real matrices -- elements of the Lie algebra-- whose eigenvalues $\lambda$ lie in the strip $\{\lambda : -\pi \le \text{Im}(\lambda) \le \pi\}$. We define the logarithm function $\log:G\to\mathcal{L}_{\mathfrak{g}}^n$ such that $\log(\mathbf{Z})$ coincides with $\Log(\mathbf{Z})$ when $\mathbf{Z}\in\mathbb{R}_+^{n\times n}$, and otherwise corresponds to any matrix $\mathbf{Y}\in\mathcal{L}_{\mathfrak{g}}^n$ that satisfies $\exp(\mathbf{Y}) = \mathbf{Z}$. This is feasible because we assume the group is exponential. Moreover, this choice should be predefined and deterministic.

With this, we can state the following important lemma.
\begin{lemma}\label{lemma:log-exp-log-equals-log}
    For all $\mathbf{Z}\in G$ and for all $r \in[0, 1]$, the following property holds:
    \begin{align*}
        \log\Bigl(\exp\bigl(r\log(\mathbf{Z})\bigr)\Bigr) = r\log(\mathbf{Z})
    \end{align*}
\end{lemma}
\begin{proof}
    The proof will be divided in three cases.
    
    \textbf{Case 1:} When $\mathbf{Z} \in \mathbb{R}_+^{n\times n}$ and $r\in[0,1]$, $\log(\mathbf{Z})=\Log(\mathbf{Z})$ and therefore $\log(\mathbf{Z})$ will lie on $\mathcal{L}^n$. Since $0\le r\le1$, then\footnote{Note that if $\lambda$ is an eigenvalue of $\mathbf{X}$, then $r\lambda$ is an eigenvalue of $r\mathbf{X}$ for any scalar $r$.} it holds that $r \log(\mathbf{Z})$ will also lie on $\mathcal{L}^n$. Moreover, since in this case, the logarithm is equal to the principal logarithm, which is invertible within its domain, it holds by definition that $\log\Bigl(\exp\bigl(r\log(\mathbf{Z})\bigr)\Bigr) = r\log(\mathbf{Z})$.

    \textbf{Case 2:} Now, let $\mathbf{Z}\notin \mathbb{R}_+^{n\times n}$ and $r\in[0, 1)$, then $\log(\mathbf{Z}) \in \mathcal{L}^n_\mathfrak{g}$. Thus $r\log(\mathbf{Z})\in\mathcal{L}^n$, where the logarithm is bijective. Therefore, the expression also holds.

    \textbf{Case 3:} Now, let $\mathbf{Z}\notin \mathbb{R}_+^{n\times n}$ and $r=1$. In this case, the expression reduces to $\log(\exp(\log(\mathbf{Z}))) = \log(\mathbf{Z})$. By the definition of $\log$, it follows that $\exp(\log(\mathbf{Z})) = \mathbf{Z}$, which ensures that the equality holds since the $\log$ function is predefined and deterministic.
\end{proof}
For the exponential Lie groups, a path $\Phi_\sigma\triangleq\Phi(\sigma, \mathbf{V}, \mathbf{W})$ can be defined as:
\begin{align}
    \Phi(\sigma, \mathbf{V}, \mathbf{W}) = \mathbf{V}\exp{\left(\log{\left(\mathbf{V}^{-1}\mathbf{W}\right)}\sigma\right)}, \label{eq:PHI-path-parameterizer-utilized-exp-of-log}
\end{align}
which is in accordance with \cref{def:PHI-path-parameterizer}. Then, an EE-distance function is defined as:
\begin{align}
    \widehat{D}(\mathbf{V}, \mathbf{W}) = \|\log{(\mathbf{V}^{-1}\mathbf{W})}\|_F.\label{eq:distance-D-hat-utilized-log-norm}
\end{align}

\begin{remark}
    The path \eqref{eq:PHI-path-parameterizer-utilized-exp-of-log} and EE-distance \eqref{eq:distance-D-hat-utilized-log-norm} reduce to the ones in \cref{ex:chainability} and \cref{ex:adriano-distance-function}, respectively, when applied to the particular case $G=T(m)$.

    Let $\mathcal{T}(\mathbf{V}) = \mathbf{v}\in\mathbb{R}^m$, $\mathcal{T}(\mathbf{W}) = \mathbf{w}\in\mathbb{R}^m$. Using the series expansion of $\log$ and $\exp$, we find that
    \begin{align}
    \begin{split}
        \mathbf{V}\exp{\bigl(\log{(\mathbf{V}^{-1}\mathbf{W})}\sigma\bigr)} 
        = \begin{bmatrix}
            \mathbf{I} & (1 - \sigma)\mathbf{v} + \sigma \mathbf{w}\\ \mathbf{0} & 1
        \end{bmatrix}.
        \end{split}
    \end{align}
    Note that $\mathcal{T}\bigl((1 - \sigma)\mathbf{V} + \sigma\mathbf{W}\bigr) = (1 - \sigma)\mathcal{T}(\mathbf{V}) + \sigma \mathcal{T}(\mathbf{W})$, the path in \cref{ex:chainability}.

    Using the series expansion of $\log$ again, $\|\log{(\mathbf{V}^{-1}\mathbf{W})}\|_F$
    $= \|\mathbf{V}^{-1}\mathbf{W} - I\|_F$. Note that $\mathbf{V}^{-1}\mathbf{W} - I$ is a matrix whose only non-zero column is the last one, equal to $[\,(\mathbf{w} - \mathbf{v})^\top\quad 0\,]^\top$, this implies that $\|\mathbf{V}^{-1}\mathbf{W} - I\|_F$$=\|\mathbf{w}-\mathbf{v}\|$, which is clearly equal to the EE-distance in \cref{ex:adriano-distance-function}. 
\end{remark}

In order to invoke \cref{thm:convergence-vector-field}, function $\widehat{D}$ in \eqref{eq:distance-D-hat-utilized-log-norm} needs to be an EE-distance (see \cref{def:distance-D-hat-arbitrary-elements}) that is left-invariant (see \cref{def:distance-left-invariant}), chainable (see \cref{def:chainable-distance}) and locally linear (see \cref{def:locallylinear}). Thus, we prove all of these properties in the following proposition.

\begin{proposition}
    Adopting the path $\Phi$ in \eqref{eq:PHI-path-parameterizer-utilized-exp-of-log}, the function $\widehat{D}$ in \eqref{eq:distance-D-hat-utilized-log-norm} is a left-invariant, chainable, and locally linear EE-distance.
\end{proposition}
\begin{proof}
    We prove each property separately. 
    
    \textbf{EE-distance}: Positive definiteness and differentiability are immediate upon inspection, and thus $\widehat{D}$ is an EE-distance.
    
    \textbf{Left-invariant}: The distance function is left-invariant since, for all $\mathbf{A} \in G$, $\widehat{D}(\mathbf{A}\mathbf{V}, \mathbf{A}\mathbf{W}) =  \|\log{(\mathbf{V}^{-1}\mathbf{A}^{-1}\mathbf{A}\mathbf{W})}\|_F$, which is clearly equal to $\widehat{D}(\mathbf{V}, \mathbf{W})$.

    \textbf{Chainable}: To prove the chainability property, we first substitute $\Phi$ by its expression \eqref{eq:PHI-path-parameterizer-utilized-exp-of-log} in \eqref{eq:distance-D-hat-utilized-log-norm}, which results in
    \begin{align}
        % \begin{split}
             \widehat{D}(\mathbf{V}, \Phi_\sigma) &= \Bigl\|\log\Bigl(\mathbf{V}^{-1}\mathbf{V}\exp\bigl(\log(\mathbf{V}^{-1}\mathbf{W})\sigma\bigr)\Bigr)\Bigr\|_F
             =\sigma\Bigl\|\log{\left(\mathbf{V}^{-1}\mathbf{W}\right)}\Bigr\|_F,
        % \end{split}
    \end{align}
    using \cref{lemma:log-exp-log-equals-log} with $\mathbf{Z}=\mathbf{V}^{-1}\mathbf{W}$ and $r=\sigma$, and the fact that $\sigma\ge0$. Now, using the fact that, by definition, $\mathbf{V}^{-1}\mathbf{W}=\exp(\log(\mathbf{V}^{-1}\mathbf{W}))$, we can express the following:
    \begin{align}
             \widehat{D}(\Phi_\sigma, \mathbf{W}) = \Bigl\|\log\Bigl(\exp\bigl(-\log(\mathbf{V}^{-1}\mathbf{W})\sigma\bigr)\exp\bigl(\log(\mathbf{V}^{-1}\mathbf{W})\bigr)\Bigr)\Bigr\|_F
    \end{align}
    Note that $\log(\mathbf{V}^{-1}\mathbf{W})$ commutes with $-\sigma\log(\mathbf{V}^{-1}\mathbf{W})$, and thus we can express the product of exponentials as the exponential of the sum of the arguments:
    \begin{align}
        \widehat{D}(\Phi_\sigma, \mathbf{W}) = \Bigl\|\log\Bigl(\exp\bigl((1-\sigma)\log(\mathbf{V}^{-1}\mathbf{W})\bigr)\Bigr)\Bigr\|_F.
    \end{align}
    Invoking \cref{lemma:log-exp-log-equals-log} with $\mathbf{Z}=\mathbf{V}^{-1}\mathbf{W}$ and $r=1-\sigma$, and using the fact that $0\le\sigma\le1$, the previous expression reduces to 
    \begin{align}
       \widehat{D}(\Phi_\sigma, \mathbf{W}) =(1-\sigma)\|\log{\left(\mathbf{V}^{-1}\mathbf{W}\right)}\|_F.
    \end{align}
    Clearly, $\widehat{D}(\mathbf{V}, \Phi_\sigma) + \widehat{D}(\Phi_\sigma, \mathbf{W}) = \|\log(\mathbf{V}^{-1}\mathbf{W})\|_F = \widehat{D}(\mathbf{V}, \mathbf{W})$. 
    
    \textbf{Locally linear}: to prove that $\widehat{D}$ is locally linear, first note that, using \cref{lemma:log-exp-log-equals-log} and the fact that $\sigma$ is non-negative, $\widehat{D}(\mathbf{V}, \Phi_\sigma) = \sigma\|\log{\left(\mathbf{V}^{-1}\mathbf{W}\right)}\|_F$, thus we have
    \begin{align}
        % \begin{split}
            \lim_{\sigma\to0^+}\frac{1}{\sigma}\widehat{D}(\mathbf{V}, \Phi_\sigma) &= \lim_{\sigma\to0^+}\frac{\sigma}{\sigma}\|\log{\left(\mathbf{V}^{-1}\mathbf{W}\right)}\|_F
            = \|\log{\left(\mathbf{V}^{-1}\mathbf{W}\right)}\|_F > 0
        % \end{split}
    \end{align}
    as long as $\mathbf{V} \not= \mathbf{W}$.
\end{proof}

\section{Explicit Construction of the EE-Distance Function in SE(3)}\label{sec:explicit-construction-SE3}
As mentioned, the group \text{SE}(3) is exponential, allowing us to utilize the construction from \cref{sec:kinematic-path-ee-dist-exp-group}. However, instead of computing $\widehat{D}(\mathbf{V},\mathbf{W}) = \|\log(\mathbf{V}^{-1}\mathbf{W})\|_F$ through a generic algorithm to compute the matrix logarithm followed by applying the Frobenius norm, the structure of the group $\text{SE}(3)$ allows a more efficient and simpler approach. The algorithm for computing $\widehat{D}(\mathbf{V},\mathbf{W})$ is as follows:

Let $\mathbf{R}_v$, $\mathbf{R}_w$, $\mathbf{p}_v$, and $\mathbf{p}_w$ denote the rotation matrices and positions of $\mathbf{V}$ and $\mathbf{W}$, respectively. Let $\mathbf{V}^{-1}\mathbf{W} = \mathbf{Z}$, and express it as 
\begin{align*}
    \mathbf{V}^{-1}\mathbf{W} = \mathbf{Z} =\begin{bmatrix}
        \mathbf{R}_v^\top\mathbf{R}_w & \mathbf{p}_w - \mathbf{p}_v \\
        \mathbf{0} & 1
    \end{bmatrix}
    = \begin{bmatrix}
        \mathbf{Q} & \mathbf{u} \\
        \mathbf{0} & 1
    \end{bmatrix}
\end{align*}

Using the power series of the matrix logarithm, it follows that
\begin{align*}
    \log\mathbf{Z} &= -\sum_{k=1}^\infty \frac{(\mathbf{I} - \mathbf{Z})^k}{k} = - (\mathbf{I} - \mathbf{Z}) - \frac{1}{2}(\mathbf{I} - \mathbf{Z})^2 - \frac{1}{3} (\mathbf{I} - \mathbf{Z})^3 - \dots\\
    &=\begin{bmatrix}
        \mathbf{I} - \mathbf{Q} & -\mathbf{u}\\\mathbf{0} & \mathbf{0}
    \end{bmatrix} -
    \frac{1}{2}\begin{bmatrix}
        (\mathbf{I} - \mathbf{Q})^2 & -(\mathbf{I} - \mathbf{Q})\mathbf{u}\\\mathbf{0} & \mathbf{0}
    \end{bmatrix} - \dots\\
    &= \begin{bmatrix}
        \log\mathbf{Q} & (\mathbf{I} - \mathbf{Q})^{-1}\log(\mathbf{Q})\mathbf{u}\\\mathbf{0} & \mathbf{0}
    \end{bmatrix} = \begin{bmatrix}
        \log\mathbf{Q} & \mathbf{X}\mathbf{u}\\\mathbf{0} & \mathbf{0}
    \end{bmatrix}.
\end{align*}
With this, we can express the EE-distance as
\begin{align}
    \begin{split}
    \widehat{D}(\mathbf{V}, \mathbf{W}) &= \|\log\mathbf{Z}\|_F = \tr\left(
        \begin{bmatrix}
            (\log\mathbf{Q})^\top\log\mathbf{Q} & (\log\mathbf{Q})^\top\mathbf{X}\mathbf{u}\\
            \mathbf{u}^\top\mathbf{X}^\top\log\mathbf{Q} & \mathbf{u}^\top\mathbf{X}^\top\mathbf{X}\mathbf{u}
        \end{bmatrix}
    \right)^\frac{1}{2}\\
        &= \sqrt{\|\log\mathbf{Q}\|_F^2 + \mathbf{u}^\top\mathbf{X}^\top\mathbf{X}\mathbf{u}} = \sqrt{\|\log\mathbf{Q}\|_F^2 + \|\bar{\mathbf{u}}\|^2},
    \end{split}
\end{align}
where $\|\bar{\mathbf{u}}\|^2 = \mathbf{u}^T\bar{\mathbf{X}}\mathbf{u}$ and $\bar{\mathbf{X}}=\mathbf{X}^\top\mathbf{X}$.

The term $\|\log\mathbf{Q}\|_F^2$ can be obtained by the following steps, using the properties derive in \cref{subsec:rotation-matrix-properties}:
\begin{enumerate}
    \item Compute $u \triangleq \frac{1}{2}\bigl(\tr(\mathbf{Q})-1\bigr)$ and $v \triangleq \frac{1}{2\sqrt{2}}\|\mathbf{Q}{-}\mathbf{Q}^{\top}\|_F$, where $u = \cos(\theta)$ and $v=\sin(\theta)$, in which $\theta \in [0, \pi]$ is the rotation angle related to $\mathbf{Q}$;
    \item Compute $\theta = \text{atan2}(v,u)$;
    \item Then, $\|\log\mathbf{Q}\|_F^2 = 2\theta^2$.
\end{enumerate}

It is also possible to derive a simple expression for $\|\bar{\mathbf{u}}\|^2$. First, we express $\bar{\mathbf{X}}$ as a function $\Phi(\mathbf{Q})$ using the Cayley-Hamilton theorem \citep[p. 63]{Chen2009}. The derivation of this function is presented in \cref{app:translation-component-SE3} and the resulting expression is:
\begin{align}
    \Phi(\mathbf{Q}) = \bar{\mathbf{X}} = (1-2\beta_0)\mathbf{I} + \beta_0\bigl(\mathbf{Q} + \mathbf{Q}^\top\bigr), \label{eq:explicit-X-bar-eedist}
\end{align}
where $\beta_0=\frac{2-2\cos\theta-\theta^2}{4(1 - \cos\theta)^2}$. Thus $\|\bar{\mathbf{u}}\|^2$ can be easily obtained from the already computed angle $\theta$ and the translation vector $\mathbf{u}$. The EE-distance function reduces to
\begin{align}
    \widehat{D}(\mathbf{V}, \mathbf{W}) = \sqrt{2\theta^2 + \mathbf{u}^T\bar{\mathbf{X}}\mathbf{u}}, \label{eq:explicit-EE-distance-SE3}
\end{align}
where $\bar{\mathbf{X}}$ is explicitly computed by \eqref{eq:explicit-X-bar-eedist}. More succinctly, the algorithm for computing $\widehat{D}(\mathbf{V}, \mathbf{W})$ is as follows:
\begin{algorithm}
    \caption{Computation of $\widehat{D}(\mathbf{V}, \mathbf{W})$ in $\text{SE}(3)$}
    \begin{algorithmic}[1]\label{alg:dhat-se3}
        \Statex \textbf{Input:} Matrices $\mathbf{V}, \mathbf{W}$
        \Statex \textbf{Output:} Distance $\widehat{D}$
        % \Statex
        
        \State Compute $\mathbf{Z} \gets \mathbf{V}^{-1}\mathbf{W}$
        \State Extract $\mathbf{Q} \gets \text{rotation part of } \mathbf{Z}$ and $\mathbf{u} \gets \text{translation part of } \mathbf{Z}$

        \State $u \gets \frac{1}{2} (\tr(\mathbf{Q}) - 1)$
        \State $v \gets \frac{1}{2\sqrt{2}} \|\mathbf{Q} - \mathbf{Q}^\top\|_F$
        \State $\theta \gets \text{atan2}(v, u)$

        \State $\beta_0 \gets \frac{2 - 2u - \theta^2}{4(1 - u)^2}$
        \State $\bar{\mathbf{X}} \gets \mathbf{I}(1 - 2\beta_0) + (\mathbf{Q} + \mathbf{Q}^\top)\beta_0$

        \State $\widehat{D} \gets \sqrt{2\theta^2 + \mathbf{u}^\top \bar{\mathbf{X}} \mathbf{u}}$
    \end{algorithmic}
\end{algorithm}

It can be shown that the result of $\|\log(\mathbf{V}^{-1}\mathbf{W})\|_F$ is independent of the choice of $\log(\mathbf{V}^{-1}\mathbf{W})$ in the edge cases where $\mathbf{V}^{-1}\mathbf{W}$ has negative eigenvalues (see the discussion in \cref{sec:kinematic-path-ee-dist-exp-group}). This is evident in the fact that \cref{alg:dhat-se3} does not include any components that require a choice to be made.

Note that $\beta_0$ is well-defined for all $\theta \in (0,\pi]$. When $\theta=0$, we just need to take the limit to obtain $\beta_0=-1/12$ (see \cref{app:discontinuity-beta-EEdist}). To identify the points of non-differentiability of $\widehat{D}$, it suffices to analyze the derivatives with the respect to the variables $\mathbf{Q}$ and $\mathbf{u}$. The analysis reveals that the only sources of non-differentiability occur when (type i) $\mathbf{Q}=\mathbf{Q}^\top$, $\mathbf{Q} \not= \mathbf{I}$ (i.e., at rotations of $\pi$ radians) or when (type ii) $\widehat{D}=0$. However, in both cases, the directional derivatives exist. Furthermore, $D_{\text{min},\mathcal{C}}$, as defined in \cref{def:distance-D-hat-arbitrary-elements}, can be taken as $\widehat{D}$ when $\theta = \pi$  and $\mathbf{u} = \mathbf{0}$, which gives $D_{\text{min},\mathcal{C}} = \sqrt{2}\pi$. Thus, when $\widehat{D} < D_{\text{min},\mathcal{C}}$, it necessarily follows that $\theta < \pi$, avoiding the non-differentiable points of type i. Additionally, when $\widehat{D} > 0$, the non-differentiable points of type ii are also avoided. Therefore, the condition $0 < \widehat{D} < \sqrt{2}\pi$ guarantees that $\widehat{D}$ is differentiable, as required in \cref{def:distance-D-hat-arbitrary-elements}.
% \begin{comment}
\section{Components computation}
Although this might not be trivial, the components of the vector field can be computed explicitly. First, note that the tangent component $\boldsymbol{\xi}_T$ relies on the nearest point in the curve and the derivative of the curve at this point. Since the curve is parametrized, we already have the equation that describes it, which implies that the derivative is also known. Thus, $\boldsymbol{\xi}_T = \invSL\bigl(\frac{d\mathbf{H}_d(s^*)}{ds}\mathbf{H}_d(s^*)^{-1}\bigr)$ is in fact computed explicitly.

To compute the normal component $\boldsymbol{\xi}_N$, we need to compute $\text{L}_{\mathbf{V}}[\widehat{D}](\mathbf{H}, \mathbf{H}_d(s^*))$. For this, two approaches exist. The first is to compute this component nummerically, by evaluating the left-hand side of \eqref{eq:Leq} for $\boldsymbol{\zeta} = \mathbf{e}_i$ with a small $\varepsilon$. The second is to compute the explicit derivative of $\widehat{D}$ with respect to $\mathbf{H}$, which we will now derive.

First, note that the EE-distance is symmetric, i.e., $\bigl\|\log(\mathbf{V}^{-1}\mathbf{W})\bigr\|_F = \bigl\|\log\bigl(\mathbf{W}^{-1}\mathbf{V}\bigr)\bigr\|_F$. Let $\mathbf{Z}=\mathbf{H}_d(s^*)^{-1}\mathbf{H}$, then using expression \eqref{eq:explicit-EE-distance-SE3}, define the following equivalent function:
\begin{align}
    \widehat{E}(\mathbf{Z}) \triangleq \widehat{D}(\mathbf{H}, \mathbf{H}_d(s^*)) = \sqrt{2\theta^2 + \mathbf{u}^\top\bar{\mathbf{X}}\mathbf{u}},
    \label{eq:explicit-EE-dista-SE3-component-Ehat}
\end{align}
for which the $\text{L}$ operator can be computed as
\begin{align}
    \Lop[\widehat{E}](\mathbf{Z}) = \frac{1}{2 \sqrt{2\theta^2 + \mathbf{u}^\top\bar{\mathbf{X}}\mathbf{u}}}\text{L}\bigl[2\theta^2 + \mathbf{u}^\top\bar{\mathbf{X}}\mathbf{u}\bigr](\mathbf{Z}),
\end{align}
which in turn can be expressed by 
\begin{align}
    \Lop[\widehat{E}](\mathbf{Z}) = \frac{4\theta\text{L}[\theta] + \Lop\bigl[\sum_{i=1}^{3}\sum_{j=1}^{3}\mathbf{u}_i\mathbf{u}_j\bar{\mathbf{X}}_{ij}\bigr](\mathbf{Z})}{2 \widehat{E}(\mathbf{Z})}. \label{eq:explicit-derivative-L-Ehat-p1}
\end{align}
The computation of the two $\Lop$ operators in \eqref{eq:explicit-derivative-L-Ehat-p1} are rather long and are shown in \cref{app:explicit-derivative-SE3}. Note however that the computation of these only rely on already computed terms and elements of $\mathbf{u}$ and $\mathbf{Q}$. For completeness, we present the expressions for these terms without the detailed derivation:
\begin{align}
    \text{L}[\theta](\mathbf{Z}) = \frac{
        -\frac{\frac{1}{2}\bigl(\tr(\mathbf{Q})-1\bigr)}{4\sqrt{3 - \tr(\mathbf{Q}^2)}}\mathbf{f} -
    \frac{\sqrt{3 - \tr(\mathbf{Q}^2)}}{2}\mathbf{g}}
    {\frac{3}{4} - \frac{1}{4}\tr(\mathbf{Q}^2)+\frac{1}{4}\bigl(\tr(\mathbf{Q})-1\bigr)^2},
    \label{eq:explicit-EE-dista-SE3-component-Ltheta}
\end{align}
where 
\begin{align}
    \mathbf{f} &= \begin{bmatrix}
        \mathbf{0}& \frac{\{\mathbf{Q}^2\}_{23}-\{\mathbf{Q}^2\}_{32}}{2} & \frac{\{\mathbf{Q}^2\}_{13} - \{\mathbf{Q}^2\}_{31}}{2} & \frac{\{\mathbf{Q}^2\}_{12}-\{\mathbf{Q}^2\}_{21}}{2}
    \end{bmatrix},\label{eq:explicit-EE-dista-SE3-component-fvec}\\
    \mathbf{g} &= \begin{bmatrix}
        \mathbf{0}& \frac{\mathbf{Q}_{23}-\mathbf{Q}_{32}}{2} & \frac{\mathbf{Q}_{13} - \mathbf{Q}_{31}}{2} & \frac{\mathbf{Q}_{12}-\mathbf{Q}_{21}}{2}
    \end{bmatrix} \label{eq:explicit-EE-dista-SE3-component-gvec}.
\end{align}
The remaining $\Lop$ operator is expressed as
\begin{align}
    \Lop\Bigl[\sum_{i=1}^{3}\sum_{j=1}^{3}\mathbf{u}_i\mathbf{u}_j\bar{\mathbf{X}}_{ij}\Bigr](\mathbf{Z}) =  \sum_{i=1}^{3}\sum_{j=1}^{3}& \Bigl(2\Lop[\mathbf{u}_{i}](\mathbf{Z})\mathbf{u}_{j}\bar{\mathbf{X}}_{ij}
     +\mathbf{u}_{i}\mathbf{u}_{j}\Lop\bigl[\bar{\mathbf{X}}_{ij}\bigr](\mathbf{Z})\Bigr),
\end{align}
where
\begin{align}
    \Lop[\mathbf{u}_{i}](\mathbf{Z}) &= \begin{bmatrix}
        \delta_{1i} & \delta_{2i} & \delta_{3i} & \{\widehat{\mathcal{S}}(\widehat{\mathbf{e}}_1)\mathbf{u}\}_i & \{\widehat{\mathcal{S}}(\widehat{\mathbf{e}}_2)\mathbf{u}\}_i & \{\widehat{\mathcal{S}}(\widehat{\mathbf{e}}_3)\mathbf{u}\}_i 
    \end{bmatrix},\label{eq:explicit-EE-dista-SE3-component-Lui}\\
    \Lop[\bar{\mathbf{X}}_{ij}](\mathbf{Z}) &= \Lop[\beta_0](\mathbf{Z})\bigl(-2\delta_{ij}+ \mathbf{Q}_{ij} + \mathbf{Q}_{ji}\bigr) + \beta_0\Bigl(\Lop[\mathbf{Q}_{ij}](\mathbf{Z}) + \Lop[\mathbf{Q}_{ji}](\mathbf{Z})\Bigr) \label{eq:explicit-EE-dista-SE3-component-Xij}
\end{align}
in which $\delta_{ij}$ is the Kronecker delta, $\widehat{\mathcal{S}}:\mathbb{R}^3\to\mathfrak{so}(3)$ is a skew-symmetric matrix, $\widehat{\mathbf{e}}_i$ is the $i^{th}$ canonical basis vector in $\mathbb{R}^3$, and
\begin{align}
    \Lop[\beta_0](\mathbf{Z}) &= \frac{\theta^2\sin(\theta) - \theta - \sin(\theta) + \bigl(\theta + \sin(\theta)\bigr) \cos(\theta)}{2 (1 - \cos(\theta))^3} \Lop[\theta](\mathbf{Z}) \label{eq:explicit-EE-dista-SE3-component-Lbeta}\\
    \Lop[\mathbf{Q}_{ij}](\mathbf{Z}) &= \begin{bmatrix}
        \mathbf{0} & \bigl\{\widehat{\mathcal{S}}(\widehat{\mathbf{e}}_1)\mathbf{Q}\bigr\}_{ij} & \bigl\{\widehat{\mathcal{S}}(\widehat{\mathbf{e}}_2)\mathbf{Q}\bigr\}_{ij} & \bigl\{\widehat{\mathcal{S}}(\widehat{\mathbf{e}}_3)\mathbf{Q}\bigr\}_{ij}
    \end{bmatrix}. \label{eq:explicit-EE-dista-SE3-component-LQij}
\end{align}

To compute $\boldsymbol{\xi}_N$ we need $\Lop_\mathbf{V}[\widehat{D}](\mathbf{H}, \mathbf{H}_d(s^*))$, however, through the chain rule, we can compute it through $\Lop[\widehat{E}](\mathbf{Z})$ as follows (see \cref{app:prop-Lop-chain-rule-SE3}):
\begin{align}
    \Lop_\mathbf{V}[\widehat{D}](\mathbf{H}, \mathbf{H}_d(s^*)) = \Lop[\widehat{E}](\mathbf{Z})\mathcal{Z}\bigl(\mathbf{H}_d(s^*)\bigr),
\end{align}
where $\mathcal{Z}\bigl(\mathbf{H}_d(s^*)\bigr)$ is computed as
\begin{align}
    \mathcal{Z}\bigl(\mathbf{H}_d(s^*)\bigr) = \begin{bmatrix}
        \mathbf{R}_d^\top(s^*) & -\mathbf{R}_d^\top(s^*)\widehat{\mathcal{S}}\bigl(\mathbf{p}_d(s^*)\bigr)\\
        \mathbf{0} & \mathbf{R}_d^\top(s^*)
    \end{bmatrix}, \label{eq:explicit-EE-dista-SE3-component-Zmap}
\end{align}
in which $\mathbf{R}_d(s^*)$ and $\mathbf{p}_d(s^*)$ are the rotation matrix and position of $\mathbf{H}_d(s^*)$, respectively. The computation of $\mathcal{Z}\bigl(\mathbf{H}_d(s^*)\bigr)$ is straightforward, and thus $\boldsymbol{\xi}_N$ can be computed explicitly. Algorithmically, the computation of $\Lop_\mathbf{V}[\widehat{D}]$ is as follows:
\begin{algorithm}[H]
    \caption{Compute the $\text{L}_\mathbf{V}[\widehat{D}]$ for $\text{SE}(3)$}
    \begin{algorithmic}[1]
    \Statex \textbf{Input:} $\mathbf{H}_d(s^*)$, $\mathbf{H}$
    \Statex \textbf{Output:} $\text{L}_\mathbf{V}[\widehat{D}]$ 
    
    \State Compute $\mathbf{Z} \gets \mathbf{H}_d(s^*)^{-1} \mathbf{H}$
    \State Extract rotation matrix $\mathbf{Q}$ and translation $\mathbf{u}$ from $\mathbf{Z}$
    \State Compute $\widehat{E}(\mathbf{Z}) \gets$  \eqref{eq:explicit-EE-dista-SE3-component-Ehat}
    \State Compute $\mathbf{f} \gets$ \eqref{eq:explicit-EE-dista-SE3-component-fvec}
    \State Compute $\mathbf{g} \gets$ \eqref{eq:explicit-EE-dista-SE3-component-gvec}
    \State Compute $\Lop[\theta] \gets$ \eqref{eq:explicit-EE-dista-SE3-component-Ltheta}
    \State Compute $\Lop[\beta_0] \gets$ \eqref{eq:explicit-EE-dista-SE3-component-Lbeta}
    \State Initialize $\Lop[\mathbf{u}^\top \bar{\mathbf{X}} \mathbf{u}] \gets 0$ 

    \For{$i \gets 1$ to $3$}
        \State Compute $\Lop[\mathbf{u}_i] \gets$ \eqref{eq:explicit-EE-dista-SE3-component-Lui}
        \For{$j \gets 1$ to $3$} 
            \State Compute $\Lop[\mathbf{Q}_{ij}] \gets$ \eqref{eq:explicit-EE-dista-SE3-component-LQij}
            \State Compute $\Lop[\bar{\mathbf{X}}_{ij}]\gets$ \eqref{eq:explicit-EE-dista-SE3-component-Xij}
            % \State Accumulate $\Lop[u^\top X u]$:
        
            \State $\Lop[\mathbf{u}^\top \bar{\mathbf{X}} \mathbf{u}] \gets$ $\Lop[\mathbf{u}^\top \bar{\mathbf{X}} \mathbf{u}]
            + 2\Lop[\mathbf{u}_i]\mathbf{u}_j \bar{\mathbf{X}}_{ij} 
            + \mathbf{u}_i \mathbf{u}_j \Lop[\bar{\mathbf{X}}_{ij}]
            $
        \EndFor
    \EndFor
    
    \State $\Lop[\widehat{E}] \gets \frac{4\theta\Lop[\theta] + \Lop[\mathbf{u}^\top \bar{\mathbf{X}} \mathbf{u}]}{\widehat{E}(\mathbf{Z})}$
    \State $\mathcal{Z}\bigl(\mathbf{H}_d(s^*)\bigr) \gets$\eqref{eq:explicit-EE-dista-SE3-component-Zmap}
    \State $\Lop_\mathbf{V}[\widehat{D}] \gets \Lop[\widehat{E}]\mathcal{Z}\bigl(\mathbf{H}_d(s^*)\bigr)$
    \end{algorithmic}
\end{algorithm}





% Thus, it suffices to show how to compute $\frac{\partial \widehat{E}}{\partial \mathbf{Z}_i}$. This can be taken as the $i^{th}$ row of the $4 \times 4$ matrix:
% \begin{equation}
%     \frac{\partial \widehat{E}}{\partial \mathbf{Z}}(\mathbf{Z}) \triangleq \left[\begin{array}{cc} \frac{\partial \widehat{E}}{\partial \mathbf{Q}}( \mathbf{Z}) & \mathbf{0} \\
%     \frac{\partial \widehat{E}}{\partial \mathbf{t}}( \mathbf{Z}) & 0 \end{array}\right]
% \end{equation}
% \noindent in which, as in Algorithm 1, $\mathbf{Q}$ and $\mathbf{t}$ are the rotation matrix and 3D translation vectors of $\mathbf{Z}$, respectively,  $ \frac{\partial \widehat{E}}{\partial \mathbf{Q}}$ the $3 \times 3$ matrix in which the entry in row $i$ and column $j$ is $\frac{\partial \widehat{E}}{\partial Q_{ji}}$ and $\frac{\partial \widehat{E}}{\partial \mathbf{t}}$ the $1 \times 3$ row vector in which the column $j$ is $\frac{\partial \widehat{E}}{\partial t_j}$. The algorithms for computing the two elements of this matrix is as follows (in which $u, v, \theta$, $\alpha$ and $\mathbf{M}$ were defined in Algorithm 1):

% \begin{itemize}
%     \item Compute 
%     \begin{eqnarray}
%        &&  \alpha'(\theta) = \frac{ \theta {+} (1 {-}\theta^2) v{-} (\theta+v)u}{2(u-1)^3} \ , \ \mathbf{N} {\triangleq} \frac{\mathbf{Q}{+}\mathbf{Q}^\top}{2}{-}\mathbf{I} \nonumber \\
%        && \frac{\partial \theta}{\partial \mathbf{Q}} = \frac{u}{4v}(\mathbf{Q}^{\top}-\mathbf{Q}) - \frac{v}{2}\mathbf{I}.
%     \end{eqnarray}
%     \item Finally, compute 
%     \begin{eqnarray}
%         && \frac{\partial \widehat{E}}{\partial \mathbf{Q}} = \frac{1}{\widehat{E}}\Bigg(\big(2\theta + \alpha' \mathbf{t}^{\top}\mathbf{N}\mathbf{t} \big) \frac{\partial \theta}{\partial \mathbf{Q}} + \alpha \mathbf{t}\mathbf{t}^{\top}\Bigg). \nonumber \\
%         && \frac{\partial \widehat{E}}{\partial \mathbf{t}} = \frac{1}{\widehat{E}} \mathbf{t}^{\top} \mathbf{M} .
%     \end{eqnarray}
% \end{itemize}
% \end{comment}