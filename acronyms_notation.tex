% !TeX root = main.tex

%----------------------------------------------------------------------------------------
%	Acronyms
%----------------------------------------------------------------------------------------
\newacronym{dof}{DoF}{Degree of Freedom}
\newacronym{uav}{UAV}{Unmanned Aerial Vehicle}
\newacronym{rgb}{RGB}{Red, Green, and Blue}
\newacronym{eedistacro}{EE}{Element-Element}
\newacronym{ecdistacro}{EC}{Element-Curve}
\newacronym{mrac}{MRAC}{Model Reference Adaptive Control}
%----------------------------------------------------------------------------------------
%	Notation
%----------------------------------------------------------------------------------------
%% Notation for Lie theory
\newnotation{G}{$G$}{Lie group}
\newnotation{GL}{$\text{GL}(n, \mathbb{R})$, $\text{GL}(n, \mathbb{C})$}{General Linear group of order $n$ over the real numbers, General Linear group of order $n$ over the complex numbers}
\newnotation{SL}{$\text{SL}(n, \mathbb{R})$, $\text{SL}(n, \mathbb{C})$}{Special Linear group of order $n$ over the real numbers, Special Linear group of order $n$ over the complex numbers}
\newnotation{SO}{$\text{O}(n)$, $\text{SO}(n)$}{Orthogonal group of order $n$, Special Orthogonal group of order $n$}
\newnotation{SE}{$\text{E}(n)$, $\text{SE}(n)$}{Euclidean group of order $n$, Special Euclidean group of order $n$}
\newnotation{Translgroup}{$\text{T}(n)$}{Translation group of order $n$. Matrix representation of $(\mathbb{R}^n, +)$}
\newnotation{isegroup}{$\text{ISE}(n)$}{Independent Special Euclidean group of order $n$, isomorphic to $\mathbb{R}^n\times\text{SO}(n)$}
\newnotation{sympgroup}{$\text{Sp}(2n)$}{Real Symplectic group of order $2n$}
\newnotation{indefort}{$\text{O}(p,q)$, $\text{SO}(p,q)$}{Indefinite Orthogonal group of signature $(p,q)$, Indefinite Special Orthogonal group of signature $(p,q)$}
\newnotation{lorentz}{$\text{O}(1, 3)$, $\text{SO}(1, 3)$, $\text{SO}^+(1, 3)$}{full Lorentz group, special Lorentz group, proper orthochronous Lorentz group}
\newnotation{g}{$\mathfrak{g}$}{Lie algebra of a Lie group $G$}
\newnotation{lieG}{$\text{Lie}(G)$}{Space of left-invariant, or right-invariant, vector fields on a Lie group $G$. Isomorphic to $\mathfrak{g}$}
\newnotation{gl}{$\mathfrak{gl}(n, \mathbb{R})$, $\mathfrak{gl}(n, \mathbb{C})$}{Lie algebras of $\text{GL}(n, \mathbb{R})$ and $\text{GL}(n, \mathbb{C})$ respectively}
\newnotation{sl}{$\mathfrak{sl}(n, \mathbb{R})$, $\mathfrak{sl}(n, \mathbb{C})$}{Lie algebras of $\text{SL}(n, \mathbb{R})$ and $\text{SL}(n, \mathbb{C})$ respectively}
\newnotation{so}{$\mathfrak{so}(n)$}{Lie algebra of $\text{O}(n)$ and $\text{SO}(n)$}
\newnotation{se}{$\mathfrak{se}(n)$}{Lie algebra of $\text{E}(n)$ and $\text{SE}(n)$}
\newnotation{transalg}{$\mathfrak{t}(n)$}{Lie algebra of $\text{T}(n)$}
\newnotation{isomorphalg}{$\mathfrak{ise}(n)$}{Lie algebra of $\text{ISE}(n)$}
\newnotation{sympalg}{$\mathfrak{sp}(2n)$}{Lie algebra of $\text{Sp}(2n)$}
\newnotation{indefortalg}{$\mathfrak{so}(p,q)$}{Lie algebra of $\text{O}(p,q)$ and $\text{SO}(p,q)$}
\newnotation{lorentzalg}{$\mathfrak{so}(1,3)$}{Lie algebra of $\text{O}(1,3)$, $\text{SO}(1,3)$ and $\text{SO}^+(1,3)$}
\newnotation{Smap}{$\mathcal{S}$}{Isomorphism that maps $\mathbb{R}^m$ to $\mathfrak{g}$}
\newnotation{invSmap}{$\mathcal{S}^{-1}$}{Inverse isomorphism that maps $\mathfrak{g}$ to $\mathbb{R}^m$, such that $\mathcal{S}^{-1}\bigl(\mathcal{S}(\boldsymbol{\xi})\bigr)=\boldsymbol{\xi}$}
\newnotation{Smapsothree}{$\widehat{\mathcal{S}}$}{Canonical isomorphism that maps $\mathbb{R}^3$ to $\mathfrak{so}(3)$}
\newnotation{SmapTn}{$\mathcal{T}$}{Isomorphism that maps $\text{T}(n)$ to $\mathbb{R}^n$}
\newnotation{euclidbasis}{$\mathbf{e}_1,\dots,\mathbf{e}_n$}{Canonical basis elements of $\mathbb{R}^n$. The columns of an $n\times n$ identity matrix}
\newnotation{rthreebasis}{$\widehat{\mathbf{e}}_1, \widehat{\mathbf{e}}_2, \widehat{\mathbf{e}}_3$}{Canonical basis elements of $\mathbb{R}^3$. The columns of a $3\times 3$ identity matrix}
\newnotation{liealgbasis}{$\mathbf{E}_1,\dots,\mathbf{E}_n$}{Basis elements of an $n$-dimensional Lie algebra}
\newnotation{liebracket}{$[\cdot, \cdot]$}{Lie bracket}
\newnotation{tangspace}{$T_pM$}{Tangent space of a manifold $M$ at a point $p$}
\newnotation{groupid}{$\iota$}{Identity element of a Lie group}
\newnotation{idcomp}{$G_\iota$}{Identity component of a Lie group $G$}
\newnotation{lefttrans}{$\mathcal{L}_\mathbf{X}$}{Left translation by $\mathbf{X}$}
\newnotation{righttrans}{$\mathcal{R}_\mathbf{X}$}{Right translation by $\mathbf{X}$}
\newnotation{groupop}{$\circ$, $\star$}{Group operations}
\newnotation{isomorph}{$\cong$}{Isomorphic. If $G$ and $H$ are isomorphic, then $G\cong H$}
\newnotation{directprod}{$\times$}{Direct product of two sets $G\times H$. Also used for the cross product of two vectors}
\newnotation{semidirectprod}{$\rtimes$}{Semidirect product of two groups $G\rtimes H$}
\newnotation{directsum}{$\oplus$}{Direct sum of two vector spaces}
\newnotation{rightarrow}{$\rightarrow$}{Indicates a mapping from the domain of a function to its codomain. The notation $f:A\to B$ means that $f$ maps elements from domain $A$ to codomain $B$}
\newnotation{mapstosymbol}{$\mapsto$}{Indicates the mapping of an element from the domain of a function to its image in the codomain. The notation $a\mapsto b$ means that $f(a)=b$ for some function $f$}
\newnotation{inclusion}{$\hookrightarrow$}{Denotes an inclusion map. The notation $f:A\hookrightarrow B$ implies that $A$ is contained within $B$ and is injectively mapped to $B$.}
%% Notation for the vector field guidance
\newnotation{Loperator}{$\text{L}[f](\mathbf{Z})$}{L operator of a function $f$ with argument $\mathbf{Z}\in G$. Maps a function $f:G\to\mathbb{R}$ into $\text{L}[f]:G\to\mathbb{R}^{1\times m}$}
\newnotation{Loppart}{$\text{L}_\mathbf{V}[\cdot](\cdot,\cdot), \text{L}_\mathbf{W}[\cdot](\cdot,\cdot)$}{L operator for a function of two variables, performing the variations on the first and second variable, respectively}
\newnotation{chainruleLop}{$\mathcal{Z}(\mathbf{Y})$}{Part of the chain rule of the L operator for left-invariant functions}
\newnotation{XIoperator}{$\Xi[\mathbf{G}]$}{Maps a differentiable function $\mathbf{G}:\mathbb{R}\to G$ into a function $\Xi[\mathbf{G}]:\mathbb{R}\to\mathbb{R}^m$}
\newnotation{twist}{$\boldsymbol{\xi}$}{Generalized twist}
\newnotation{vectorfield}{$\Psi$}{Vector field that guides a system along a curve in a Lie group}
\newnotation{tangentcomp}{$\boldsymbol{\xi}_T$}{Tangent component of the vector field $\Psi$}
\newnotation{normalcomp}{$\boldsymbol{\xi}_N$}{Normal component of the vector field $\Psi$}
\newnotation{compgains}{$k_N, k_T$}{Gains for the normal and tangent components of the vector field $\Psi$, respectively. Both are continuous scalar functions}
\newnotation{eedist}{$\widehat{D}(\mathbf{V}, \mathbf{W})$}{Element-Element (EE-) distance between two Lie group elements $\mathbf{V}$ and $\mathbf{W}$}
\newnotation{ecdistance}{$D(\mathbf{V})$}{Element-Curve (EC-) distance between a Lie group element $\mathbf{V}$ and a curve $\mathcal{C}$}
\newnotation{problemset}{$\mathcal{P}$}{Set of all points that are either non-unique solutions or non-differentiable points of the EE-distance function}
\newnotation{curve}{$\mathcal{C}$}{A curve in a Lie group}
\newnotation{systemstate}{$\mathbf{H}$}{The system state, an element of a Lie group}
\newnotation{curveparam}{$\mathbf{H}_d(s)$}{A curve parameterized by $s$ in a Lie group}
\newnotation{nearestpoint}{$\mathbf{H}_d(s^*)$}{The nearest point on a curve $\mathcal{C}$ to a system state $\mathbf{H}$}
\newnotation{curvetwist}{$\boldsymbol{\xi}_d(s)$}{The curve twist, $\boldsymbol{\xi}_d(s)=\Xi[\mathbf{H}_d](s)$}
\newnotation{discretecurve}{$\mathbf{H}_d[i]$}{A discretized curve in a Lie group at the $i^{th}$ point}
\newnotation{discretenearpoint}{$\mathbf{H}_d[i^*]$}{The nearest point on a discretized curve to a system state $\mathbf{H}$}
\newnotation{discretecurvederivative}{$\mathbf{H}_d'[i]$}{The derivative of a discretized curve at the $i^{th}$ point}
\newnotation{liepath}{$\Phi$}{A continuous and differentiable path in a Lie group}


\newnotation{posechi}{$\boldsymbol{\chi}$}{Pose of a rigid body represented as a tuple in $\mathbb{R}^3\times\text{SO}(3)$}
\newnotation{heunsvari}{$\underline{x}$}{Intermediate variable of $x$ in the Heun's method}
\newnotation{timederiv}{$\dot{x}$}{Time derivative of a time-dependent variable $x$}

\newnotation{diagop}{$\diag(\cdot)$}{Maps a vector into a diagonal matrix}
\newnotation{blkdiagop}{$\blkdiag(\cdot,\dots,\cdot)$}{Maps a set of matrices into a block-diagonal matrix}
\newnotation{elementabs}{$\abs(\cdot)$}{Element-wise absolute value of a vector}

\newnotation{rotop}{$\rot(\cdot)$}{Extracts the rotation matrix from an element of $\text{SE}(3)$ or $\text{ISE}(3)$}
\newnotation{transop}{$\trans(\cdot)$}{Extracts the translation vector from an element of $\text{SE}(3)$ or $\text{ISE}(3)$}
\newnotation{idmatrix}{$\mathbf{I}_n, \mathbf{I}$}{Identity matrix of dimension $n$. If the dimension is clear from the context, then $\mathbf{I}$ is used}
\newnotation{Ipq}{$\mathbf{I}_{p,q}$}{A block-diagonal matrix containing a $p\times p$ identity matrix, and a $q\times q$ identity matrix multiplied by $-1$}
\newnotation{zeromatrix}{$\mathbf{0}_{n\times m}, \mathbf{0}$}{Zero matrix, or vector, of dimension $n\times m$. If the dimension is clear from the context, then $\mathbf{0}$ is used}
\newnotation{matrixcol}{$\mathbf{A}_i, \{\mathbf{A}\}_i$}{The $i^{th}$ column of a matrix $\mathbf{A}$}
\newnotation{matrixcolrow}{$\mathbf{A}_{ij},\{\mathbf{A}\}_{ij}$}{The element at the $i^{th}$ row and $j^{th}$ column of a matrix $\mathbf{A}$}
\newnotation{vecelement}{$\mathbf{v}_i, \{\mathbf{v}\}_i$}{The $i^{th}$ element of a vector $\mathbf{v}$}
\newnotation{tranpose}{$\cdot^\top$}{Transpose of a matrix or vector}
\newnotation{Log}{$\Log(\cdot)$}{Principal matrix logarithm}
\newnotation{log}{$\log(\cdot)$}{Matrix logarithm and scalar natural logarithm}
\newnotation{exp}{$\exp(\cdot)$}{Matrix exponential and scalar exponential}
\newnotation{tr}{$\tr(\cdot)$}{Matrix trace}
\newnotation{norm}{$\|\cdot\|$}{Euclidean norm}
\newnotation{frobnorm}{$\|\cdot\|_F$}{Frobenius norm}

\newnotation{realnumbers}{$\mathbb{R}$}{Set of real numbers}
\newnotation{Rplus}{$\mathbb{R}_+$}{Set of non-negative real numbers}
\newnotation{Rn}{$\mathbb{R}^n$}{Euclidean space of dimension $n$. The elements are $n$-dimensional column-vectors}
\newnotation{Rnn}{$\mathbb{R}^{n\times n}$}{Set of $n\times n$ real matrices}
\newnotation{Rnnplus}{$\mathbb{R}^{n\times n}_+$}{Set of $n\times n$ real matrices with no negative real eigenvalues}
\newnotation{complexnumbers}{$\mathbb{C}$}{Set of complex numbers}
\newnotation{Cnn}{$\mathbb{C}^{n\times n}$}{Set of $n\times n$ complex matrices}
\newnotation{sphereSn}{$\mathbb{S}^n$}{Unit sphere in $\mathbb{R}^{n+1}$}

\newnotation{innerprod}{$\langle\cdot,\cdot\rangle$}{Inner product between two vectors}
\newnotation{smallo}{$o(\cdot)$}{Small-o notation}
\newnotation{gradient}{$\nabla$}{Gradient operator}
\newnotation{krondelta}{$\delta_{ij}$}{Kronecker delta}
\newnotation{emptyset}{$\emptyset$}{Empty set}

