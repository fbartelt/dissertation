% !TeX root = Template Latex - Apresentacao - IFSP - SBV.tex
\section{Theoretical Background}

\begin{frame}
    \frametitle{Vector field in Euclidean space}
    \begin{columns}[c]
        \begin{column}{7cm}
            The vector field strategy in \citet{Rezende2022} is based on a parametric curve representation and is characterized by:
            \begin{itemize}
                \item A distance function $D$;
                \item $\dot{D}=\nabla D\boldsymbol{\xi}=-\boldsymbol{\xi}_N^\top\boldsymbol{\xi}$;
                \item Tangent component depends only on the curve;
                \item Normal and tangent components are orthogonal;
                \item Absence of local minima outside the curve;
                \item Vector field $\Psi(\mathbf{h})=k_N(D)\boldsymbol{\xi}_N(\mathbf{h}) + k_T(D)\boldsymbol{\xi}_T(\mathbf{h})$.
            \end{itemize}
            
        \end{column}
        \begin{column}{8cm}
           \begin{figure}[ht!]
            \centering
            \def\svgwidth{\linewidth}
            {\footnotesize\import{../figures/}{plotly_vf2.pdf_tex}}
        \end{figure}
        \end{column}
    \end{columns}
\end{frame}
\begin{frame}{Lie Groups and Lie algebras}
    \begin{columns}[c]
        \begin{column}{0.5\linewidth}
            \begin{exampleblock}{Lie group $G$}
                Manifolds with group structure. The group operation and inverse map are continuous and smooth.
                \linebreak

                E.g.: set of rotation matrices $\text{SO}(3)$.
            \end{exampleblock}
            \begin{exampleblock}{Lie algebra $\mathfrak{g}$}
                Tangent space of $G$ at the identity.
                \linebreak

                E.g.: skew-symmetric matrices for $\text{SO}(3)$.
            \end{exampleblock}
        \end{column}
        \begin{column}{0.5\linewidth}
            \begin{exampleblock}{Exponential map}
                Maps elements of $\mathfrak{g}$ to $G$. For matrix Lie groups: $\exp(\mathbf{A})=\sum_{i=0}^\infty \frac{\mathbf{A}^n}{n!} = \mathbf{X} \in G$.

                Not always surjective.
            \end{exampleblock}
            \begin{exampleblock}{$\mathcal{S}$ map}
                Linear map from $\mathbb{R}^m$ to $\mathfrak{g}$, relating velocities to tangent space elements.
                \linebreak

                E.g.: angular velocities $\to$ skew-symmetric matrices in $\mathfrak{so}(3)$.
            \end{exampleblock}    
        \end{column}
    \end{columns}
\end{frame}
