\chapter{In-depth derivations for the EE-Distance in SE(3)}\label{app:rodrigues-formula}
In this chapter we provide all the derivations for the EE-Distance in SE(3) and its properties. We start by deriving the Rodrigues formula for rotation matrices in $\text{SO}(3)$ and showing some important properties of rotation matrices. This is done aiming at providing a self-contained text. Although we will not treat the EE-distance directly here, showing only the key properties used to obtain it, the reader will easily make the necessary connections to understand the derivations in \cref{sec:explicit-construction-SE3}

\section{Deriving Rodrigues' Rotation Formula}
In this section we derive the Rodrigues formula through a Lie group approach. This will be useful for obtaining important properties of rotation matrices. Let $\mathbf{R}\in\text{SO}(3)$ be a rotation matrix. Since $\text{SO}(3)$ is exponential group, it follows that it is the exponential of some Lie algebra element $\SL[\boldsymbol{\omega}]\in\mathfrak{so}(3)$, i.e.
\begin{align}
    \mathbf{R} = \exp\bigl(\SL[\boldsymbol{\omega}]\bigr).
\end{align}

We start by exploring powers of the Lie algebra element. First, let $\boldsymbol{\omega} = [\omega_1\ \omega_2\ \omega_3]^\top$, thus we can express the squared Lie algebra element as
\begin{align}
    \SL[\boldsymbol{\omega}]^2 = \begin{bmatrix}
        0 & -\omega_3 & \omega_2\\
        \omega_3 & 0 & -\omega_1 \\
        -\omega_2 & \omega_1 & 0
    \end{bmatrix}^2 = 
    \begin{bmatrix}
        -\omega_3^2-\omega_2^2 & \omega_2\omega_1 & \omega_3\omega_1\\
        \omega_1\omega_2 & -\omega_3^2-\omega_1^2 & \omega_3\omega_2 \\
        \omega_1\omega_3 & \omega_2\omega_3 & -\omega_2^2-\omega_1^2
    \end{bmatrix}.
\end{align}
Furthermore, if we let $\theta=\sqrt{\omega_1^2+\omega_2^2+\omega_3^2}$, and the following matrix
\begin{align}
    \mathbf{B} = \begin{bmatrix}
        \omega_1^2 & \omega_2\omega_1 & \omega_3\omega_1\\
        \omega_1\omega_2 & \omega_2^2 & \omega_3\omega_2 \\
        \omega_1\omega_3 & \omega_2\omega_3 & \omega_3^2
    \end{bmatrix},
\end{align}
it is clear that we can express $\SL[\boldsymbol{\omega}]^2$ as
\begin{align}
    \SL[\boldsymbol{\omega}]^2 = -\theta^2\mathbf{I} + \mathbf{B}.
\end{align}
With this equation we can also express
\begin{align}
    \SL[\boldsymbol{\omega}]^3 = \SL[\boldsymbol{\omega}]^2\SL[\boldsymbol{\omega}] = -\theta^2\SL[\boldsymbol{\omega}] + \mathbf{B}\SL[\boldsymbol{\omega}].
\end{align}
Note that
\begin{align}
    \SL[\boldsymbol{\omega}]\mathbf{B} &= \begin{bmatrix}
        0 & -\omega_3 & \omega_2\\
        \omega_3 & 0 & -\omega_1 \\
        -\omega_2 & \omega_1 & 0
    \end{bmatrix}\begin{bmatrix}
        \omega_1^2 & \omega_2\omega_1 & \omega_3\omega_1\\
        \omega_1\omega_2 & \omega_2^2 & \omega_3\omega_2 \\
        \omega_1\omega_3 & \omega_2\omega_3 & \omega_3^2
    \end{bmatrix}\\
    &= \begin{bmatrix}
        -\omega_3\omega_1\omega_2 + \omega_2\omega_1\omega_3 & -\omega_3\omega_2^2+\omega_2^2\omega_3 & -\omega_3^2\omega_2 + \omega_2\omega_3^2\\
        \omega_3\omega_1^2-\omega_1^2\omega_3 & \omega_3\omega_2\omega_1 - \omega_1\omega_2\omega_3 & \omega_3^2\omega_1 - \omega_1\omega_3^2\\
        -\omega_2\omega_1^2 + \omega_1^2\omega_2 & -\omega_2\omega_1 + \omega_1\omega_2^2 & -\omega_2\omega_3\omega_1 + \omega_1\omega_2\omega_3
    \end{bmatrix}\\
    &=\mathbf{0},
\end{align}
which is also true for $\mathbf{B}\SL[\boldsymbol{\omega}]$, since $(\mathbf{B}\SL[\boldsymbol{\omega}])^\top=-\SL[\boldsymbol{\omega}]\mathbf{B}$. Finally, we can express
\begin{align}
    \SL[\boldsymbol{\omega}]^3 &= -\theta^2\SL[\boldsymbol{\omega}],\\
    \SL[\boldsymbol{\omega}]^4 &= \SL[\boldsymbol{\omega}]^3\SL[\boldsymbol{\omega}] = \theta^4\mathbf{I} - \theta^2\mathbf{B}.
\end{align}
Thus, by induction, we can express that for any $k>0$:
\begin{align}
    \SL[\boldsymbol{\omega}]^{4k+1} &= \theta^{4k}\SL[\boldsymbol{\omega}],\\
    \SL[\boldsymbol{\omega}]^{4k+2} &= -\theta^{4k+2}\mathbf{I} + \theta^{4k}\mathbf{B},\\
    \SL[\boldsymbol{\omega}]^{4k+3} &= -\theta^{4k+2}\SL[\boldsymbol{\omega}],\\
    \SL[\boldsymbol{\omega}]^{4k+4} &= \theta^{4k+4}\mathbf{I} - \theta^{4k+2}\mathbf{B}.
\end{align}

Using the power series of the matrix exponential, the expression for the rotation matrix $\mathbf{R}$ is
\begin{align}
    \begin{split}
    &\exp\bigl(\SL[\boldsymbol{\omega}]\bigr) = \sum_{k=0}^\infty\frac{\SL[\boldsymbol{\omega}]^k}{k!}
    = \mathbf{I} + \SL[\boldsymbol{\omega}] + \frac{\SL[\boldsymbol{\omega}]^2}{2!} + \frac{\SL[\boldsymbol{\omega}]^3}{3!} + \frac{\SL[\boldsymbol{\omega}]^4}{4!} + \dots\\
    &= \mathbf{I} + \SL[\boldsymbol{\omega}] + \frac{-\theta^2\mathbf{I} + \mathbf{B}}{2!} - \frac{\theta^2\SL[\boldsymbol{\omega}]}{3!} + \frac{\theta^4\mathbf{I} - \theta^2\mathbf{B}}{4!} + \frac{\theta^4\SL[\boldsymbol{\omega}]}{5!} + \frac{-\theta^6\mathbf{I}+\theta^4\mathbf{B}}{6!} +\dots\\
    &= \biggl(1 - \frac{\theta^2}{2!} + \frac{\theta^4}{4!} + \dots\biggr)\mathbf{I} + \biggl(1 - \frac{\theta^2}{3!} + \frac{\theta^4}{5!} + \dots\biggr)\SL[\boldsymbol{\omega}] + \biggl(\frac{1}{2!} - \frac{\theta^2}{4!} + \frac{\theta^4}{6!}+\dots\biggr)\mathbf{B}.
    \end{split}
\end{align}
Manipulating the previous expression and using the power series of sine and cosine functions, we can express the rotation matrix as
\begin{align}
    \mathbf{R} &= \biggl(\sum_{i=0}^\infty (-1)^k\frac{\theta^{2k}}{(2k)!}\biggr)\mathbf{I} + \frac{1}{\theta}\biggl(\sum_{i=0}^\infty (-1)^k\frac{\theta^{2k+1}}{(2k+1)!}\biggr)\SL[\boldsymbol{\omega}] + \frac{1}{\theta^2}\biggl(1 - \sum_{i=0}^\infty (-1)^k\frac{\theta^{2k}}{(2k)!}\biggr)\mathbf{B}\\
    &= \cos(\theta)\mathbf{I} + \frac{\sin\theta}{\theta}\SL[\boldsymbol{\omega}] + \frac{1-\cos\theta}{\theta^2}\mathbf{B}.
\end{align}
Using the fact that $\SL[\boldsymbol{\omega}]^2 = -\theta^2\mathbf{I} + \mathbf{B}$, we achieve the most common form of the Rodrigues formula:
\begin{align}
    \mathbf{R} = \mathbf{I} + \frac{\sin\theta}{\theta}\SL[\boldsymbol{\omega}] + \frac{1-\cos\theta}{\theta^2}\SL[\boldsymbol{\omega}]^2.
\end{align}
\section{Rotation matrix properties}\label{subsec:rotation-matrix-properties}
The Rodrigues formula allows us to derive important properties of rotation matrices. 
\subsection{Cosine of the angle}
For instance, we can show that the trace of a rotation matrix is equal to $1+2\cos\theta$. This can be shown by taking the trace of the Rodrigues formula:
\begin{align}
    \tr(\mathbf{R}) &= \tr\biggl(\cos(\theta)\mathbf{I} + \frac{\sin\theta}{\theta}\SL[\boldsymbol{\omega}] + \frac{1-\cos\theta}{\theta^2}\mathbf{B}\biggr)\\
    &= 3\cos\theta + \frac{1-\cos\theta}{\theta^2}\tr(\mathbf{B})\\
    &= 3\cos\theta + 1-\cos\theta\\
    &= 1+2\cos\theta.
\end{align}

\subsection{Sine of the angle}
It is also possible to find a relation for the $\sin\theta$ by computing $\|\mathbf{R} - \mathbf{R}^\top\|_F$. First, we compute the subtraction using Rodrigues formula:
\begin{align}
    % \begin{split}
        \mathbf{R} - \mathbf{R}^\top &= \cos(\theta)\mathbf{I} + \frac{\sin\theta}{\theta}\SL[\boldsymbol{\omega}] + \frac{1-\cos\theta}{\theta^2}\mathbf{B} - \cos(\theta)\mathbf{I} + \frac{\sin\theta}{\theta}\SL[\boldsymbol{\omega}] - \frac{1-\cos\theta}{\theta^2}\mathbf{B}\nonumber\\
        &= \frac{2\sin\theta}{\theta}\SL[\boldsymbol{\omega}].
    % \end{split}
\end{align}
Then, we compute the Frobenius norm of the subtraction:
\begin{align}
    % \begin{split}
    \|\mathbf{R} - \mathbf{R}^\top\|_F &= \sqrt{\tr\biggl(\biggl(\frac{2\sin\theta}{\theta}\SL[\boldsymbol{\omega}]\biggr)^\top\biggl(\frac{2\sin\theta}{\theta}\SL[\boldsymbol{\omega}]\biggr)\biggr)}\\
    &= \sqrt{\tr\biggl(-\frac{4(\sin\theta)^2}{\theta^2}\Bigl( -\theta^2\mathbf{I} + \mathbf{B}\Bigr)\biggr)}\\
    &=\sqrt{ 12(\sin\theta)^2 - 4(\sin\theta)^2}
    \implies \sin\theta = \pm\frac{1}{2\sqrt{2}}\|\mathbf{R} - \mathbf{R}^\top\|_F
% \end{split}
\end{align}

\subsection{Eigenvalues of the rotation matrix}
We can find the eigenvalues of a rotation matrix $\mathbf{R}$ by noting that the eigenvalues of $\exp\bigl(\SL[\boldsymbol{\omega}]\bigr)$ are the exponentials of the eigenvalues of $\SL[\boldsymbol{\omega}]$. Thus, we find the eigenvalues $\lambda$ of $\SL[\boldsymbol{\omega}]$ by computing the characteristic polynomial:
\begin{align}
    \det\bigl(\lambda\mathbf{I} - \SL[\boldsymbol{\omega}]\bigr)&=\begin{vmatrix}
        \lambda & \omega_3 & -\omega_2\\
        -\omega_3 & \lambda & \omega_1\\
        \omega_2 & -\omega_1 & \lambda
    \end{vmatrix} = \lambda^3 + \lambda(\omega_1^2 + \omega_2^2 + \omega_3^2) = \lambda(\lambda^2 + \theta^2)=0\\
    \implies \lambda&\in\{0, i\theta, -i\theta\}.
\end{align}
Thus, the eigenvalues of $\mathbf{R}$ are $e^0=1$, $e^{i\theta}$, and $e^{-i\theta}$.

\subsection{Frobenius norm of the logarithm of a rotation matrix}
The Frobenius norm of the logarithm of a rotation matrix can be computed by noting that, since $\mathbf{R} = \exp\bigl(\SL[\boldsymbol{\omega}]\bigr)$, then $\SL[\boldsymbol{\omega}] = \log(\mathbf{R})$. Thus, we can compute the Frobenius norm of the logarithm of a rotation matrix as
\begin{align}
    \|\log(\mathbf{R})\|_F^2=\|\SL[\boldsymbol{\omega}]\|_F^2 &= \tr\bigl(\SL[\boldsymbol{\omega}]^\top\SL[\boldsymbol{\omega}]\bigr) = \tr\bigl(-\SL[\boldsymbol{\omega}]^2\bigr) = \tr\bigl(\theta^2\mathbf{I} - \mathbf{B}\bigr) = 2\theta^2.
\end{align}

\section{Translation Component of the SE(3) EE-Distance}\label{app:translation-component-SE3}
Our objective in this section is to derive the function that allows us to compute the translation component of the EE-Distance in SE(3) (see \cref{sec:explicit-construction-SE3}). Let $\mathbf{X}=(\mathbf{I} - \mathbf{Q})^{-1}\log(\mathbf{Q})$, where $\mathbf{Q}\in\text{SO}(3)$ is a rotation matrix. Furthermore, let $\bar{\mathbf{X}}=\mathbf{X}^\top\mathbf{X}$. The derivations here will then allow a simple expression for $\mathbf{u}^\top\bar{\mathbf{X}}\mathbf{u}$ for any $\mathbf{u}\in\mathbb{R}^3$.

We start by expressing $\bar{\mathbf{X}}$ as as a function $\Phi(\mathbf{Q})$:
\begin{align}
    \bar{\mathbf{X}} &= \mathbf{X}^T\mathbf{X} = (\log\mathbf{Q})^\top\bigl((\mathbf{I} - \mathbf{Q})^{-1}\bigr)^\top(\mathbf{I} - \mathbf{Q})^{-1}\log\mathbf{Q}\\
    &= -\log\mathbf{Q}(\mathbf{I} - \mathbf{Q}^\top)^{-1}(\mathbf{I} - \mathbf{Q})^{-1}\log\mathbf{Q}\\
    &=-\log\mathbf{Q}\bigl((\mathbf{I} - \mathbf{Q})(\mathbf{I} - \mathbf{Q}^\top)\bigr)^{-1}\log\mathbf{Q}\\
    &=-\log\mathbf{Q}(2\mathbf{I}-(\mathbf{Q}+\mathbf{Q}^\top))^{-1}\log\mathbf{Q} \label{eq:appendix-Xbar-translation-dhat}\\
    &= \Phi(\mathbf{Q}),
\end{align}
where we used the fact\footnote{The fact that $(\log\mathbf{Q})^\top = \log(\mathbf{Q}^\top) = -\log\mathbf{Q}$ can be easily seen through the power series of the matrix logarithm or from the fact that the logarithm of a rotation matrix is a skew-symmetric matrix.} that $(\log\mathbf{Q})^\top=-\log\mathbf{Q}$. Now, from Cayley-Hamilton theorem \citep[p. 63]{Chen2009}, this function can be expressed by the characteristic polynomial of $\mathbf{Q}$:
\begin{align}
    \Phi(\mathbf{Q}) = \beta_0\mathbf{Q}^{-1} + \beta_1\mathbf{I} + \beta_2\mathbf{Q},
\end{align}
for which the coefficients $\beta_i$ can be found by means of the eigenvalues of $\mathbf{Q}$. These coefficients can be found by evaluating the function $\Phi$ at each eigenvalue and comparing to \eqref{eq:appendix-Xbar-translation-dhat}. Since the eigenvalues $\lambda_i$ of $\mathbf{Q}$ are $\{1, e^{i\theta}, e^{-i\theta}\},\theta\in[0,\pi]$, we must solve the following equaiton for each $\lambda_i$:
\begin{align}
    \beta_0\lambda^{-1} + \beta_1 + \beta_2\lambda = -\log\lambda_i(2-(\lambda+\lambda^{-1}))^{-1}\log\lambda.
\end{align}
Thus, for each eigenvalue, we have:
\begin{align}
    % \begin{cases}
        \beta_0 + \beta_1 + \beta_2 &= \frac{-\log(1)^2}{2 - (1+1)}=1,\quad\text{ for }\lambda = 1 \label{eq:appendix-cayley-lambda1}\\
        \beta_0 e^{-i\theta} + \beta_1 + \beta_2 e^{i\theta} &= \frac{\theta^2}{2 - (e^{i\theta}+ e^{-i\theta})},\quad\text{ for }\lambda = e^{i\theta} \label{eq:appendix-cayley-lambdaeitheta}\\ 
        \beta_0 e^{i\theta} + \beta_1 + \beta_2 e^{-i\theta} &= \frac{\theta^2}{2 - (e^{-i\theta}+ e^{i\theta})},\quad\text{ for }\lambda = e^{i\theta}. \label{eq:appendix-cayley-lambdaMINUSeitheta}
    % \end{cases}
\end{align}
Subtracting \eqref{eq:appendix-cayley-lambdaMINUSeitheta} from \eqref{eq:appendix-cayley-lambdaeitheta} and multiplying by $\frac{1}{2i}$ results in
\begin{align}
    \frac{1}{2i}\beta_2(e^{i\theta}-e^{-i\theta}) +\frac{1}{2i}\beta_0(e^{-i\theta}-e^{i\theta}) &= \biggl(\frac{\theta^2}{2 - (e^{i\theta}+ e^{-i\theta})} - \frac{\theta^2}{2 - (e^{-i\theta}+ e^{i\theta})}\biggr)\frac{1}{2i}\\
    \beta_2\sin\theta -\beta_0\sin\theta &= 0\\
    \implies \beta_0&=\beta_2 \label{eq:appendix-cayley-beta0}
\end{align}
Now, summing \eqref{eq:appendix-cayley-lambdaeitheta} and \eqref{eq:appendix-cayley-lambdaMINUSeitheta}, multiplying by $\frac{1}{2}$, and using $\beta_0=\beta_2$, results in:
\begin{align}
    \beta_1+\frac{1}{2}\beta_2(e^{i\theta}+e^{-i\theta}) +\frac{1}{2}\beta_0(e^{-i\theta}+e^{i\theta}) &= \biggl(\frac{\theta^2}{2 - (e^{i\theta}+ e^{-i\theta})} + \frac{\theta^2}{2 - (e^{-i\theta}+ e^{i\theta})}\biggr)\frac{1}{2}\\
    \beta_1+2\beta_0\cos\theta  &= \frac{\theta^2}{2 - 2\cos\theta}\\
    \implies \beta_1&=\frac{\theta^2}{2 - 2\cos\theta} - 2\beta_0\cos\theta. \label{eq:appendix-cayley-beta1}
\end{align}
Finally, substituting \eqref{eq:appendix-cayley-beta0} and \eqref{eq:appendix-cayley-beta1} into \eqref{eq:appendix-cayley-lambda1}, results in
\begin{align}
    1 &= \frac{\theta^2}{2 - 2\cos\theta} - 2\beta_0\cos\theta + 2\beta_0 \\
    \implies \beta_0 &= \frac{2-2\cos\theta-\theta^2}{4(1 - \cos\theta)^2}. 
\end{align}
Thus, we can express
\begin{align}
    \Phi(\mathbf{Q}) = \bar{\mathbf{X}} = (1-2\beta_0)\mathbf{I} + \beta_0\bigl(\mathbf{Q} + \mathbf{Q}^\top\bigr), \label{eq:appendix-explicit-X-bar-eedist}
\end{align}
since $\beta_1=1-\beta_0-\beta_1=1-2\beta_0$. Thus, since we can easily obtain the angle $\theta$ from the rotation matrix, the expression $\mathbf{u}^\top\bar{\mathbf{X}}\mathbf{u}$ does not need to envolve the computation of matrix logarithms.

\subsection{Discontinuity analysis}\label{app:discontinuity-beta-EEdist}
The expression in \eqref{eq:appendix-explicit-X-bar-eedist} depends on the coefficient $\beta_0$, which in turn has a discontinuity when $\theta=0$, since the denominator would be zero. Our objective in this section is to show that the limit of $\beta_0$ as $\theta\to 0$ exists and can be used when $\theta\approx0$. First note that when $\theta\to 0$, $\beta_0$ is indeterminate, thus we apply L'Hôpital's rule:
\begin{align}
    \lim_{\theta\to0}\beta_0=\lim_{\theta\to0} \frac{2-2\cos\theta-\theta^2}{4(1 - \cos\theta)^2}
    &= \lim_{\theta\to0} \frac{2\sin\theta-2\theta}{8(1 - \cos\theta)\sin\theta},
\end{align}
which is again indeterminate. Applying L'Hôpital's rule again, we find
\begin{align}
    \lim_{\theta\to0} \frac{2\sin\theta-2\theta}{8(1 - \cos\theta)\sin\theta}
    &= \lim_{\theta\to0} \frac{2\cos\theta-2}{8(1 - \cos\theta)\cos\theta +8(\sin\theta)^2},
\end{align}
which is still indeterminate. Applying L'Hôpital's rule one more time, we find
\begin{align}
    \lim_{\theta\to0} \frac{2\cos\theta-2}{8(1 - \cos\theta)\cos\theta +8(\sin\theta)^2} =
    \lim_{\theta\to0}\frac{-2\sin\theta}{-8(1 - \cos\theta)\sin\theta +24\cos\theta\sin\theta},
\end{align}
which remains indeterminate. Applying L'Hôpital's rule one last time, we find
\begin{align}
    \lim_{\theta\to0}\beta_0&=\lim_{\theta\to0}\frac{-2\sin\theta}{-8(1 - \cos\theta)\sin\theta +24\cos\theta\sin\theta} \\
    &=
    \lim_{\theta\to0}\frac{-2\cos\theta}{-8(1 - \cos\theta)\cos\theta +24(\cos\theta)^2-32(\sin\theta)^2} = -\frac{1}{12}.
\end{align}
Thus, when $\theta\approx0$, we can use $\beta_0=-\frac{1}{12}$ in \eqref{eq:appendix-explicit-X-bar-eedist}.

\section{Explicit derivative of the EE-Distance in SE(3)}\label{app:explicit-derivative-SE3}
In this section, our objective is to derive the explicit expression for the derivative of the EE-Distance $\widehat{D}$ in $\text{SE}(3)$ of \cref{sec:explicit-construction-SE3}. This will allow us to compute the normal vector $\boldsymbol{\xi}_N$ in \cref{def:normal-vector} explictly. First, we will introduce two operators that will be used in this derivation. Let $\rot(\cdot)$ be an operator that extracts the rotation matrix of a group element, such that $\rot(\mathbf{Z})=\mathbf{Q}\in\text{SO}(3)$. Additionally, let $\trans(\cdot)$ be an operator that extracts the translation vector of a group element, such that $\trans(\mathbf{Z})=\mathbf{u}\in\mathbb{R}^3$. Thus, the EE-distance is expressed as
\begin{align}
    \widehat{E}(\mathbf{Z}) = \sqrt{2\theta^2 + \trans(\mathbf{Z})^\top\bar{\mathbf{X}}\trans(\mathbf{Z})},
\end{align}
where $\theta$ is a function dependent on $\rot(\mathbf{Z})$. We will also need a clear distinction between the canonical isomorphisms for $\mathfrak{se}(3)$ and $\mathfrak{so}(3)$. Let $\widehat{\mathcal{S}}:\mathbb{R}^3\to\mathfrak{so}(3)$ be the canonical isomorphism of $\mathfrak{so}(3)$, and $\SL:\mathbb{R}^6\to\mathfrak{se}(3)$ be the canonical isomorphism of $\mathfrak{se}(3)$. This also requires the notation $\widehat{\mathbf{e}}_i$ for the canonical basis of $\mathbb{R}^3$ and $\mathbf{e}_i$ for the canonical basis of $\mathbb{R}^6$.

The $\Lop$ operator of the EE-distance with respect to $\mathbf{Z}$ is then expressed as
% To compute $\boldsymbol{\xi}_N$ according to \cref{def:normal-vector}, we need to compute $\text{L}_{\mathbf{V}}[\widehat{D}](\mathbf{H}, \mathbf{H}_d(s^*))$. First, note that the EE-distance is symmetric, i.e., $\bigl\|\log(\mathbf{V}^{-1}\mathbf{W})\bigr\|_F = \bigl\|\log\bigl(\mathbf{W}^{-1}\mathbf{V}\bigr)\bigr\|_F$. Let $\mathbf{Z}=\mathbf{H}_d(s^*)^{-1}\mathbf{H}$, then using expression \eqref{eq:explicit-EE-distance-SE3}, the $\text{L}$ operator can be computed as
\begin{align}
    \begin{split}
        \text{L}[\widehat{E}](\mathbf{Z}) &=  \frac{1}{2\sqrt{2\theta^2 + \trans(\mathbf{Z})^\top\bar{\mathbf{X}}\trans(\mathbf{Z})}} \text{L}\bigl[2\theta^2 + \trans(\mathbf{Z})^\top\bar{\mathbf{X}}\trans(\mathbf{Z})\bigr](\mathbf{Z}) \\
        &= \frac{1}{2\sqrt{2\theta^2 + \trans(\mathbf{Z})^\top\bar{\mathbf{X}}\trans(\mathbf{Z})}} \Bigl(\text{L}\bigl[2\theta^2\bigr](\mathbf{Z}) + \text{L}\bigl[\trans(\mathbf{Z})^\top\bar{\mathbf{X}}\trans(\mathbf{Z})\bigr](\mathbf{Z})\Bigr),\!\!\!\!\! \label{eq:explicit-derivative-SE3-LdhatZ-firststep}
    \end{split}
\end{align}
where $\theta, \mathbf{u}$ and $\bar{\mathbf{X}}$ are all functions of $\mathbf{Z}$. We now analyze the two terms separately. Starting with the first term, we have
\begin{align}
    \text{L}\bigl[2\theta^2\bigr](\mathbf{Z}) = 4\theta \text{L}[\theta](\mathbf{Z}).
\end{align}
Now, as shown in \cref{sec:explicit-construction-SE3}, $\theta$ is expressed as $\text{atan2}(\sin\theta, \cos\theta)$, however both derivatives of $\text{atan2}$ and $\tan^{-1}$ coincide. The $\text{L}$ operator of $\theta$ is then expressed as
\begin{align}
    \text{L}[\theta](\mathbf{Z}) = \frac{1}{\Biggl(\frac{\frac{1}{2\sqrt{2}}\bigl\|\rot(\mathbf{Z}) - \rot(\mathbf{Z})^\top\bigr\|_F}{\frac{1}{2}\Bigl(\tr\bigl(\rot(\mathbf{Z})\bigr)-1\Bigr)}\Biggr)^2+1} \Lop\left[\rule{0em}{9mm}\frac{\frac{1}{2\sqrt{2}}\bigl\|\rot(\mathbf{Z}) - \rot(\mathbf{Z})^\top\bigr\|_F}{\frac{1}{2}\Bigl(\tr\bigl(\rot(\mathbf{Z})\bigr)-1\Bigr)}\right](\mathbf{Z}), \label{eq:explicit-derivative-SE3-LthetaZ}
\end{align}
where the expressions for $\sin\theta$ and $\cos\theta$ from \cref{subsec:rotation-matrix-properties} have been used. Focusing for now only on the denominator of \eqref{eq:explicit-derivative-SE3-LthetaZ}, let $\mathbb{R}^6\ni\boldsymbol{\zeta} = [\mathbf{v}^\top\ \boldsymbol{\omega}^\top]^\top$ with $\mathbf{v}, \boldsymbol{\omega}\in\mathbb{R}^3$. Then, clearly
\begin{align}
    \rot\Bigl(\exp\bigl(\varepsilon\SL[\boldsymbol{\zeta}]\bigr)\mathbf{Z}\Bigr) = \exp\bigl(\varepsilon\widehat{\mathcal{S}}(\boldsymbol{\omega})\bigr)\rot(\mathbf{Z}). 
\end{align}
Using the implicit definition of $\Lop$ through derivatives, we have
\begin{align}
   \begin{split}
     \Lop\biggl[\frac{1}{2}\Bigl(\tr\bigl(\rot(\mathbf{Z})\bigr)-1\Bigr)\biggr](\mathbf{Z})\boldsymbol{\zeta} &= \frac{1}{2}\frac{d}{d\varepsilon}\Biggl(\tr\biggl(\rot\Bigl(\exp\bigl(\varepsilon\SL[\boldsymbol{\zeta}]\bigr)\mathbf{Z}\Bigr)\biggr)\Biggr)\Biggr|_{\varepsilon=0}\\
     = \frac{1}{2}\frac{d}{d\varepsilon}\Biggl(\tr\biggl(\exp\bigl(\varepsilon\widehat{\mathcal{S}}(\boldsymbol{\omega})\bigr)\rot(\mathbf{Z})\biggr)\Biggr)\Biggr|_{\varepsilon=0}
     &= \frac{1}{2}\tr\Biggl(\frac{d}{d\varepsilon}\biggl(\exp\bigl(\varepsilon\widehat{\mathcal{S}}(\boldsymbol{\omega})\bigr)\rot(\mathbf{Z})\biggr)\biggr|_{\varepsilon=0}\Biggr)\\
     = \frac{1}{2}\tr\Biggl(\biggl(\widehat{\mathcal{S}}(\boldsymbol{\omega})\exp\bigl(\varepsilon\widehat{\mathcal{S}}(\boldsymbol{\omega})\bigr)\rot(\mathbf{Z})\biggr)\biggr|_{\varepsilon=0}\Biggr)
     &= \frac{1}{2}\tr\bigl(\widehat{\mathcal{S}}(\boldsymbol{\omega})\rot(\mathbf{Z})\bigr),
   \end{split}
\end{align}
which in turn implies that 
\begin{align}
    \Lop\biggl[\frac{1}{2}\Bigl(\tr\bigl(\rot(\mathbf{Z})\bigr)-1\Bigr)\biggr](\mathbf{Z}) = \begin{bmatrix}
        \mathbf{0}& \frac{\tr\bigl(\widehat{\mathcal{S}}(\widehat{\mathbf{e}}_1)\rot(\mathbf{Z})\bigr)}{2} & \frac{\tr\bigl(\widehat{\mathcal{S}}(\widehat{\mathbf{e}}_2)\rot(\mathbf{Z})\bigr)}{2} & \frac{\tr\bigl(\widehat{\mathcal{S}}(\widehat{\mathbf{e}}_3)\rot(\mathbf{Z})\bigr)}{2}
    \end{bmatrix}.
\end{align}
If we denote $\rot(\mathbf{Z})_{ij}$ the element in the $i^{th}$ row and $j^{th}$ column of $\rot(\mathbf{Z})$, then this expression further reduces to
\begin{align}
    \Lop\biggl[\frac{1}{2}\Bigl(\tr\bigl(\rot(\mathbf{Z})\bigr)-1\Bigr)\biggr](\mathbf{Z}) = \begin{bmatrix}
        \mathbf{0}& \frac{\rot(\mathbf{Z})_{23}-\rot(\mathbf{Z})_{32}}{2} & \frac{\rot(\mathbf{Z})_{13} - \rot(\mathbf{Z})_{31}}{2} & \frac{\rot(\mathbf{Z})_{12}-\rot(\mathbf{Z})_{21}}{2}
    \end{bmatrix}.
\end{align}

Switching our attention back to the numerator of \eqref{eq:explicit-derivative-SE3-LthetaZ}, note that
\begin{align}
    &\bigl\|\rot(\mathbf{Z}) - \rot(\mathbf{Z})^\top\bigr\|_F = 
    \tr\Bigl(\bigl(\rot(\mathbf{Z}) - \rot(\mathbf{Z})^\top\bigr)^\top\bigl(\rot(\mathbf{Z}) - \rot(\mathbf{Z})^\top\bigr)\Bigr)^{\frac{1}{2}}\\
    &= \tr\Bigl(\rot(\mathbf{Z})^\top\rot(\mathbf{Z})-\rot(\mathbf{Z})^\top\rot(\mathbf{Z})^\top + \rot(\mathbf{Z})\rot(\mathbf{Z})^\top - \rot(\mathbf{Z})\rot(\mathbf{Z})\Bigr)^{\frac{1}{2}}\\
    &= \sqrt{2\tr(\mathbf{I}) - 2\tr\bigl(\rot(\mathbf{Z})\rot(\mathbf{Z})\bigr)} = \sqrt{6 - 2\tr\bigl(\rot(\mathbf{Z})\rot(\mathbf{Z})\bigr)},
\end{align}
using properties of the trace. From this, the $\text{L}$ operator of the numerator is given by
\begin{align}
    \Lop\biggl[\frac{1}{2\sqrt{2}}\bigl\|\rot(\mathbf{Z}) - \rot(\mathbf{Z})^\top\bigr\|_F\biggr](\mathbf{Z}) = \frac{-2}{4\sqrt{2}\sqrt{6 - 2\tr\bigl(\rot(\mathbf{Z})^2\bigr)}}\Lop\Bigl[\tr\bigl(\rot(\mathbf{Z})^2\bigr)\Bigr](\mathbf{Z}).
\end{align}
Using the already computed $\text{L}$ operator of the trace, this reduces to
\begin{align}
    \Lop\biggl[\frac{1}{2\sqrt{2}}\bigl\|\rot(\mathbf{Z}) - \rot(\mathbf{Z})^\top\bigr\|_F\biggr](\mathbf{Z}) = -\frac{1}{4\sqrt{3 - \tr\bigl(\rot(\mathbf{Z})^2\bigr)}}\begin{bmatrix}
        \mathbf{0}& \alpha_1 & \alpha_2 & \alpha_3
    \end{bmatrix},
\end{align}
where 
\begin{align}
    \begin{split}
        \alpha_1 =& -\rot(\mathbf{Z})_{12}\rot(\mathbf{Z})_{31} + \rot(\mathbf{Z})_{13}\rot(\mathbf{Z})_{21} + \rot(\mathbf{Z})_{22}\rot(\mathbf{Z})_{23}  \\
        & - \rot(\mathbf{Z})_{22}\rot(\mathbf{Z})_{32} + \rot(\mathbf{Z})_{23}\rot(\mathbf{Z})_{33} - \rot(\mathbf{Z})_{32}\rot(\mathbf{Z})_{33},\\
        \alpha_2 =& -\rot(\mathbf{Z})_{11}\rot(\mathbf{Z})_{13} + \rot(\mathbf{Z})_{11}\rot(\mathbf{Z})_{31} - \rot(\mathbf{Z})_{12}\rot(\mathbf{Z})_{23} \\
        &- \rot(\mathbf{Z})_{13}\rot(\mathbf{Z})_{33} + \rot(\mathbf{Z})_{21}\rot(\mathbf{Z})_{32} + \rot(\mathbf{Z})_{31}\rot(\mathbf{Z})_{33},
        \\
        \alpha_3 =& \rot(\mathbf{Z})_{11}\rot(\mathbf{Z})_{12} - \rot(\mathbf{Z})_{11}\rot(\mathbf{Z})_{21} + \rot(\mathbf{Z})_{12}\rot(\mathbf{Z})_{22} \\
        &+ \rot(\mathbf{Z})_{13}\rot(\mathbf{Z})_{32} - \rot(\mathbf{Z})_{21}\rot(\mathbf{Z})_{22} - \rot(\mathbf{Z})_{23}\rot(\mathbf{Z})_{31}.
    \end{split}
\end{align}

Now, note that the quotient rule for derivatives also applies to the $\Lop$ operator. If we use the following equivalences:
\begin{align}
    \mathbf{f} &= \begin{bmatrix}
        \mathbf{0}& \alpha_1 & \alpha_2 & \alpha_3
    \end{bmatrix},\\
    \mathbf{g} &= \begin{bmatrix}
        \mathbf{0}& \frac{\rot(\mathbf{Z})_{23}-\rot(\mathbf{Z})_{32}}{2} & \frac{\rot(\mathbf{Z})_{13} - \rot(\mathbf{Z})_{31}}{2} & \frac{\rot(\mathbf{Z})_{12}-\rot(\mathbf{Z})_{21}}{2}
    \end{bmatrix},
\end{align}
then we can express $\Lop[\theta]$ as follows
% \begin{align}
%     \begin{split}
%         \Lop\left[\rule{0em}{9mm}\frac{\frac{1}{2\sqrt{2}}\bigl\|\rot(\mathbf{Z}) - \rot(\mathbf{Z})^\top\bigr\|_F}{\frac{1}{2}\Bigl(\tr\bigl(\rot(\mathbf{Z})\bigr)-1\Bigr)}\right]\!\!(\mathbf{Z}) &= \frac{\Lop\biggl[\frac{1}{2\sqrt{2}}\bigl\|\rot(\mathbf{Z}) - \rot(\mathbf{Z})^\top\bigr\|_F\biggr](\mathbf{Z})\frac{1}{2}\Bigl(\tr\bigl(\rot(\mathbf{Z})\bigr)-1\Bigr)
%         }{\frac{1}{4}\Bigl(\tr\bigl(\rot(\mathbf{Z})\bigr)-1\Bigr)^2}\\
%         &- \frac{\Lop\biggl[\frac{1}{2}\Bigl(\tr\bigl(\rot(\mathbf{Z})\bigr)-1\Bigr)\biggr](\mathbf{Z})\frac{1}{2\sqrt{2}}\bigl\|\rot(\mathbf{Z}) - \rot(\mathbf{Z})^\top\bigr\|_F}{\frac{1}{4}\Bigl(\tr\bigl(\rot(\mathbf{Z})\bigr)-1\Bigr)^2}
%     \end{split}
% \end{align}
\begin{align}
    \text{L}[\theta](\mathbf{Z}) = \frac{
        -\frac{\frac{1}{2}\Bigl(\tr\bigl(\rot(\mathbf{Z})\bigr)-1\Bigr)}{4\sqrt{3 - \tr\bigl(\rot(\mathbf{Z})^2\bigr)}}\mathbf{f} -
    \frac{\sqrt{3 - \tr\bigl(\rot(\mathbf{Z})^2\bigr)}}{2}\mathbf{g}}
    {\frac{3}{4} - \frac{1}{4}\tr\bigl(\rot(\mathbf{Z})^2\bigr)+\frac{1}{4}\Bigl(\tr\bigl(\rot(\mathbf{Z})\bigr)-1\Bigr)^2} 
\end{align}

To finally determine $\Lop[\widehat{E}]$, we still need to determine $\Lop\bigl[\trans(\mathbf{Z})^\top\bar{\mathbf{X}}\trans(\mathbf{Z})\bigr]$. Since $\Lop$ is defined for scalar functions, we first express $\trans(\mathbf{Z})^\top\bar{\mathbf{X}}\trans(\mathbf{Z})$ element wise, which results in
\begin{align}
    \trans(\mathbf{Z})^\top\bar{\mathbf{X}}\trans(\mathbf{Z}) = \sum_{i=1}^{3}\sum_{j=1}^{3} \trans(\mathbf{Z})_{i}\trans(\mathbf{Z})_{j}\bar{\mathbf{X}}_{ij}.
\end{align}
The $\Lop$ operator of this expression is given by
\begin{align}
    \begin{split}
        \Lop\bigl[\trans(\mathbf{Z})^\top\bar{\mathbf{X}}\trans(\mathbf{Z})\bigr](\mathbf{Z}) = \sum_{i=1}^{3}\sum_{j=1}^{3}& \Bigl(2\Lop\bigl[\trans(\mathbf{Z})_{i}\bigr](\mathbf{Z})\trans(\mathbf{Z})_{j}\bar{\mathbf{X}}_{ij} \\
        & +\trans(\mathbf{Z})_{i}\trans(\mathbf{Z})_{j}\Lop\bigl[\bar{\mathbf{X}}_{ij}\bigr](\mathbf{Z})\Bigr),
    \end{split}
\end{align}
since $\bar{\mathbf{X}}$ is symmetric. The $\Lop$ operator of $\trans(\mathbf{Z})_{i}$ is given by its implicit expression
\begin{align}
    \begin{split}
        \Lop\bigl[\trans(\mathbf{Z})_{i}\bigr](\mathbf{Z}) \boldsymbol{\zeta}&= 
        \frac{d}{d\varepsilon}\biggl(\trans\Bigl(\exp\bigl(\varepsilon\SL[\boldsymbol{\zeta}]\bigr)\mathbf{Z}\Bigr)_{i}\biggr)\biggr|_{\varepsilon=0}\!\!\! = 
        \trans\biggl(\frac{d}{d\varepsilon}\Bigl\{\exp\bigl(\varepsilon\SL[\boldsymbol{\zeta}]\bigr)\mathbf{Z}\Bigr\}_{i}\biggr|_{\varepsilon=0}\biggr) \!\!\!\!\!\!\\
        &= \trans\biggl(\Bigl\{\SL[\boldsymbol{\zeta}]\exp\bigl(\varepsilon\SL[\boldsymbol{\zeta}]\bigr)\mathbf{Z}\Bigr\}_{i}\biggr|_{\varepsilon=0}\biggr)
        = \trans\Bigl(\bigl\{\SL[\boldsymbol{\zeta}]\mathbf{Z}\bigr\}_{i}\Bigr)\\
        &= \bigl\{\widehat{\mathcal{S}}(\boldsymbol{\omega})\trans(\mathbf{Z}) + \mathbf{v}\bigr\}_{i},
    \end{split}
\end{align}
which in turn implies that
\begin{align}
    \Lop\bigl[\trans(\mathbf{Z})_{i}\bigr](\mathbf{Z}) = \begin{bmatrix}
        \delta_{1i} & \delta_{2i} & \delta_{3i} & \{\widehat{\mathcal{S}}(\widehat{\mathbf{e}}_1)\trans(\mathbf{Z})\}_i & \{\widehat{\mathcal{S}}(\widehat{\mathbf{e}}_2)\trans(\mathbf{Z})\}_i & \{\widehat{\mathcal{S}}(\widehat{\mathbf{e}}_3)\trans(\mathbf{Z})\}_i
    \end{bmatrix},
\end{align}
where $\delta_{ij}$ is the Kronecker delta, i.e., $\delta_{ij}=1 \iff i=j$ and $\delta_{ij}=0$ otherwise. Regarding $\Lop[\bar{\mathbf{X}}_{ij}]$ we have
\begin{align}
    \Lop[\bar{\mathbf{X}}_{ij}](\mathbf{Z})= \Lop[\beta_0](\mathbf{Z})\bigl(-2\delta_{ij}+ \rot(\mathbf{Z})_{ij} + \rot(\mathbf{Z})_{ji}\bigr) + \beta_0\Bigl(\Lop[\rot(\mathbf{Z})_{ij}](\mathbf{Z}) + \Lop[\rot(\mathbf{Z})_{ij}^\top](\mathbf{Z})\Bigr)
\end{align}
Expressing the $\Lop$ operator of  each element in the $i^{th}$ row and $j^{th}$ column of $\rot(\mathbf{Z})$, we have
\begin{align}
    \Lop[\rot(\mathbf{Z})_{ij}](\mathbf{Z})\boldsymbol{\zeta} &=  \left.\frac{d}{d\varepsilon}\left(\biggl\{\rot\Bigl(\exp\bigl(\varepsilon\SL[\boldsymbol{\zeta}]\bigr)\mathbf{Z}\Bigr)\biggr\}_{ij}\right)\right|_{\varepsilon=0}\\
    &=  \frac{d}{d\varepsilon}\biggl(\Bigl\{\exp\bigl(\varepsilon\widehat{\mathcal{S}}(\boldsymbol{\omega})\bigr)\rot(\mathbf{Z})\Bigr\}_{ij}\biggr)\biggr|_{\varepsilon=0}\\
    &=\bigl\{\widehat{\mathcal{S}}(\boldsymbol{\omega})\rot(\mathbf{Z})\bigr\}_{ij},
\end{align}
which implies that
\begin{align}
    \Lop[\rot(\mathbf{Z})_{ij}](\mathbf{Z}) &= \begin{bmatrix}
        \mathbf{0} & \bigl\{\widehat{\mathcal{S}}(\widehat{\mathbf{e}}_1)\rot(\mathbf{Z})\bigr\}_{ij} & \bigl\{\widehat{\mathcal{S}}(\widehat{\mathbf{e}}_2)\rot(\mathbf{Z})\bigr\}_{ij} & \bigl\{\widehat{\mathcal{S}}(\widehat{\mathbf{e}}_3)\rot(\mathbf{Z})\bigr\}_{ij}
    \end{bmatrix}.
\end{align}
Clearly, the $\Lop$ operator of each element in the $i^{th}$ row and $j^{th}$ column of $\rot(\mathbf{Z})^\top$ is directly given by $\Lop[\rot(\mathbf{Z})_{ij}](\mathbf{Z})$:
\begin{align}
    \Lop[\rot(\mathbf{Z})^\top_{ij}](\mathbf{Z}) = \Lop[\rot(\mathbf{Z})_{ji}](\mathbf{Z}).
\end{align}
The last term remaining is $\Lop[\beta_0](\mathbf{Z})$, which is given by
\begin{align}
    \Lop[\beta_0](\mathbf{Z}) &= \frac{d\beta_0}{d\theta} \Lop[\theta](\mathbf{Z}),\\
    &= \frac{\theta^2\sin(\theta) - \theta - \sin(\theta) + \bigl(\theta + \sin(\theta)\bigr) \cos(\theta)}{2 (1 - \cos(\theta))^3} \Lop[\theta](\mathbf{Z}).
\end{align}
Now, we have all the ingredients necessary to compute
\begin{align}
    \Lop[\widehat{E}](\mathbf{Z}) = \frac{1}{2\widehat{E}(\mathbf{V}, \mathbf{W})} \Bigl(4\theta\text{L}[\theta](\mathbf{Z}) + \text{L}\bigl[\trans(\mathbf{Z})^\top\bar{\mathbf{X}}\trans(\mathbf{Z})\bigr](\mathbf{Z})\Bigr)
\end{align}